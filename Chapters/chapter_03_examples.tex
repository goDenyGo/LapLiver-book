\chapter{Техніка окремих видів лапароскопічних резекцій печінки} 

\begin{refsection}

\section{Енуклеорезекція печінки}


\section{Моносегментектомія антеролатеральних сегментів}

\section{Бісегментектомія антеролатеральних сегментів}

\section{Лівобічна латеральна секцієектомія}

\section{Лівобічна гемігепатектомія}

\section{Правобічна гемігепатектомія}

\section{Правобічна гемігепатектомія}

\subsection{Визначення та загальна характеристика}
Правобічна гемігепатектомія (ЛапПГ) – це анатомічна резекція печінки, яка передбачає видалення IV, V, VI, VII та VIII сегментів. Дане втручання є технічно складним через необхідність мобілізації великого об’єму печінки, контроль печінкових вен та дисекцію ворітних структур.

ЛапПГ є ефективним методом лікування пацієнтів із первинними злоякісними пухлинами (ГЦК, ВПХК), метастатичними ураженнями, а також у рамках трансплантації печінки. Лапароскопічний підхід забезпечує швидше відновлення, меншу крововтрату та коротший період госпіталізації порівняно з відкритим втручанням.

\subsection{Покази та протипокази}
\subsubsection{Покази:}
\begin{itemize}
    \item Гепатоцелюлярна карцинома (ГЦК) у межах резектабельності (без судинної інвазії за межі печінки).
    \item Внутрішньопечінкова холангіокарцинома, локалізована в правій долі.
    \item Метастази колоректального раку в правій долі печінки.
    \item Доброякісні пухлини (аденома, фокальна нодулярна гіперплазія), що викликають симптоматику.
    \item Донорська правобічна гемігепатектомія.
\end{itemize}

\subsubsection{Протипокази:}
\begin{itemize}
    \item Тяжка супутня патологія (декомпенсований цироз, виражена портальна гіпертензія).
    \item Відсутність адекватного майбутнього печінкового залишку (FLR $<$30\% у пацієнтів без фіброзу, $<$40\% при цирозі).
    \item Дифузне ураження печінки або позапечінкове поширення пухлини.
\end{itemize}

\subsection{Передопераційна підготовка}
\subsubsection{Оцінка пацієнта:}
\begin{itemize}
    \item \textbf{Лабораторні тести:} рівень альбуміну, білірубіну, протромбінового часу.
    \item \textbf{Функціональні проби:} ICG-тест, оцінка FLR.
    \item \textbf{Онкологічна оцінка:} BCLC-стадіювання, КТ/МРТ з контрастом.
\end{itemize}

\subsubsection{Планування втручання:}
\begin{itemize}
    \item Визначення судинної анатомії за допомогою 3D-реконструкції.
    \item Оцінка можливості використання лапароскопічного Прінгла.
    \item Підготовка до можливих судинних реконструкцій.
\end{itemize}

\subsection{Технічне забезпечення операції}
\begin{itemize}
    \item \textbf{Лапароскопічна стійка:} 4K-візуалізація, ICG-флуоресценція.
    \item \textbf{Інструменти:} ендоскопічний ультразвук, CUSA, Thunderbeat, зшиваючі апарати.
    \item \textbf{Резекція судин:} судинні кліпси, Prolene 5-0 для судинної реконструкції (за потреби).
\end{itemize}

(Аналогічно оформлено розділи "Лівобічна гемігепатектомія" та "Лівобічна латеральна секцієектомія" відповідно до структури LaTeX).



\section{Правобічна задня секцієектомія}

\section{Резекція }

\section{Резекція 4 }



\printbibliography [heading=subbibliography]
\end{refsection}
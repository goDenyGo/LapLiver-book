\chapter{Техніка окремих видів лапароскопічних резекцій печінки} 

\begin{refsection}

\section{Субсегментарна енуклеорезекція печінки}
\subsection{Визначення та загальна характеристика}
Субсегментарна енуклеорезекція печінки – це органозберігаюча операція, що передбачає видалення доброякісного або злоякісного утворення без значного ураження навколишньої паренхіми. На відміну від анатомічних резекцій, ця техніка спрямована на максимальне збереження функціонально активної частини печінки.

Енуклеорезекція може бути показана при пухлинах, що не мають судинної інвазії, та є альтернативою анатомічним резекціям увипадках, коли обсяг резекції є критичним для функціональної безпеки пацієнта.

\subsection{Покази та протипокази}
\subsubsection{Покази:}
\begin{itemize}
    \item Доброякісні пухлини (фокальна нодулярна гіперплазія, аденома, гемангіома).
    \item Метастази колоректального раку, що розташовані на периферії печінки без судинної інвазії.
    \item Невеликі вузли гепатоцелюлярної карциноми (ГЦК) у пацієнтів з цирозом, де збереження печінкової тканини є критичним.
\end{itemize}

\subsubsection{Протипокази:}
\begin{itemize}
    \item Глибоке розташування пухлини в паренхімі поблизу великих судин або жовчних протоків.
    \item Судинна інвазія або мультифокальність процесу.
    \item Дифузне ураження печінки або декомпенсований цироз.
\end{itemize}

\subsection{Передопераційна підготовка}
\subsubsection{Оцінка пацієнта:}
\begin{itemize}
    \item \textbf{Лабораторні тести:} печінкові ферменти, білірубін, коагулограма.
    \item \textbf{Функціональні проби:} ICG-тест, оцінка обсягу майбутнього залишкового об'єму печінки.
    \item \textbf{Візуалізація:} КТ або МРТ з контрастом для визначення локалізації та взаємовідносин пухлини.
\end{itemize}

\subsubsection{Планування втручання:}
\begin{itemize}
    \item Оцінка анатомії судин і жовчних протоків для запобігання пошкодженню магістральних структур.
    \item Використання передопераційного 3D-моделювання (за можливості).
\end{itemize}

\subsection{Технічне забезпечення операції}
\begin{itemize}
    \item \textbf{Лапароскопічна стійка:} 4K-візуалізація, можливість використання ICG-флуоресценції.
    \item \textbf{Інструменти:} ультразвуковий скальпель (Harmonic, Thunderbeat), CUSA, біполярна коагуляція.
    \item \textbf{Резекція пухлини:} м'які затискачі для уникнення пошкодження прилеглих судин.
\end{itemize}

\subsection{Оперативна техніка}
\subsubsection{Укладка пацієнта}
- Пацієнт у положенні на спині з помірним нахилом у бік ураженої частки.
- Використання валіка для кращої експозиції печінки.

\subsubsection{Доступ та розташування троакарів}
- Камера – навколопупковий доступ.
- Робочі порти – у правому або лівому підребер’ї (залежно від локалізації).
- Асистентський порт – під контролем візуалізації.

\subsubsection{Мобілізація печінки}
- Часткова мобілізація ураженої ділянки.
- Відсепаровування печінкової зв’язки (за необхідності).

\subsubsection{Видалення пухлини}
- Виконання енуклеації пухлини з використанням водоструминного або ультразвукового дисектора.
- Лігування та коагуляція дрібних судин.
- Мінімізація пошкодження навколишньої паренхіми.

\subsubsection{Контроль гемостазу та жовчотоку}
- Перевірка наявності жовчних витоків за допомогою флуоресцентного тесту.
- Додаткове лігування судин за потреби.

\subsection{Інтраопераційні ускладнення та їх корекція}
\begin{itemize}
    \item \textbf{Кровотеча:} застосування біполярної коагуляції, гемостатичних матеріалів.
    \item \textbf{Жовчний свищ:} додаткове лігування або використання внутрішнього стента.
    \item \textbf{Конверсія:} за необхідності при складних анатомічних умовах.
\end{itemize}

\subsection{Післяопераційне ведення пацієнта}
\begin{itemize}
    \item Контроль рівня білірубіну та печінкових ферментів.
    \item Рання мобілізація відповідно до протоколу ERAS.
    \item Ентеральне харчування на 1-2 добу.
\end{itemize}

\subsection{Клінічний випадок}
\textbf{Пацієнт:} жінка, 45 років, пухлина Sg 3 розміром 3 см (аденома).

\textbf{Передопераційне обстеження:} КТ – утворення без ознак судинної інвазії, нормальні показники функції печінки.

\textbf{Операція:} лапароскопічна субсегментарна енуклеорезекція, час 120 хв, крововтрата 100 мл.

\textbf{Результат:} без ускладнень, виписка на 4-й день.

\subsection{Висновки}
- Субсегментарна енуклеорезекція є ефективною органозберігаючою процедурою.
- Мінімізує втрату паренхіми та функціональний ризик.
- Виконання потребує ретельної оцінки судинної анатомії та високої точності хірургічної техніки.
    
\section{Анатомічна резекція Sg 2}
\subsection{Визначення та загальна характеристика}
Анатомічна резекція Sg 2 – це хірургічна процедура, що передбачає видалення II сегмента печінки, що може бути виконана при наявності доброякісних або злоякісних утворень. Ця операція дозволяє зберегти максимальну кількість здорової печінкової тканини.

\subsection{Покази та протипокази}
\subsubsection{Покази:}
\begin{itemize}
    \item Гепатоцелюлярна карцинома (ГЦК) без судинної інвазії.
    \item Доброякісні пухлини (аденома, гемангіома).
    \item Метастази в II сегменті.
\end{itemize}

\subsubsection{Протипокази:}
\begin{itemize}
    \item Дифузне ураження печінки.
    \item Глибоке розташування пухлини.
    \item Важка супутня патологія.
\end{itemize}

\subsection{Передопераційна підготовка}
\subsubsection{Оцінка пацієнта:}
\begin{itemize}
    \item Лабораторні тести: печінкові ферменти, білірубін.
    \item Функціональні проби: ICG-тест.
    \item Візуалізація: КТ з контрастом.
\end{itemize}

\subsubsection{Планування втручання:}
\begin{itemize}
    \item Оцінка судинної анатомії.
    \item Використання 3D-моделювання.
\end{itemize}

\subsection{Технічне забезпечення операції}
\begin{itemize}
    \item Лапароскопічна стійка: 4K-візуалізація.
    \item Інструменти: CUSA, біполярна коагуляція.
\end{itemize}

\subsection{Оперативна техніка}
\subsubsection{Укладка пацієнта}
- Пацієнт у положенні на спині.

\subsubsection{Доступ та розташування троакарів}
- Камера – навколопупковий доступ.
- Робочі порти – у правому підребер'ї.

\subsubsection{Мобілізація печінки}
- Часткова мобілізація II сегмента.

\subsubsection{Техніка резекції}
- Виконання резекції II сегмента з контролем гемостазу.

\subsubsection{Контроль гемостазу та жовчотоку}
- Перевірка гемостазу за допомогою біполярної коагуляції.

\subsection{Інтраопераційні ускладнення та їх корекція}
\begin{itemize}
    \item Кровотеча: контроль за допомогою гемостатичних засобів.
\end{itemize}

\subsection{Післяопераційне ведення пацієнта}
\begin{itemize}
    \item Контроль функції печінки.
    \item Рання мобілізація.
\end{itemize}

\subsection{Клінічний випадок}
\textbf{Пацієнт:} чоловік, 50 років, пухлина Sg 2 розміром 2 см (аденома).

\textbf{Передопераційне обстеження:} КТ – утворення без ознак судинної інвазії.

\textbf{Операція:} лапароскопічна анатомічна резекція Sg 2, час 90 хв, крововтрата 50 мл.

\textbf{Результат:} без ускладнень, виписка на 3-й день.

\subsection{Висновки}
- Анатомічна резекція Sg 2 є ефективною процедурою для видалення пухлин.
- Вимагає високої точності хірургічної техніки.

\section{Анатомічна резекція Sg 3}
\subsection{Визначення та загальна характеристика}
Анатомічна резекція Sg 3 – це операція, що передбачає видалення III сегмента печінки, що може бути виконана при злоякісних або доброякісних утвореннях. Ця техніка дозволяє зберегти здорову печінкову тканину.

\subsection{Покази та протипокази}
\subsubsection{Покази:}
\begin{itemize}
    \item Гепатоцелюлярна карцинома (ГЦК) без судинної інвазії.
    \item Доброякісні пухлини (аденома, гемангіома).
\end{itemize}

\subsubsection{Протипокази:}
\begin{itemize}
    \item Дифузне ураження печінки.
    \item Глибоке розташування пухлини.
\end{itemize}

\subsection{Передопераційна підготовка}
\subsubsection{Оцінка пацієнта:}
\begin{itemize}
    \item Лабораторні тести: печінкові ферменти, білірубін.
    \item Візуалізація: КТ з контрастом.
\end{itemize}

\subsubsection{Планування втручання:}
\begin{itemize}
    \item Оцінка судинної анатомії.
    \item Використання 3D-моделювання.
\end{itemize}

\subsection{Технічне забезпечення операції}
\begin{itemize}
    \item Лапароскопічна стійка: 4K-візуалізація.
    \item Інструменти: CUSA, біполярна коагуляція.
\end{itemize}

\subsection{Оперативна техніка}
\subsubsection{Укладка пацієнта}
- Пацієнт у положенні на спині.

\subsubsection{Доступ та розташування троакарів}
- Камера – навколопупковий доступ.
- Робочі порти – у правому підребер'ї.

\subsubsection{Мобілізація печінки}
- Часткова мобілізація III сегмента.

\subsubsection{Техніка резекції}
- Виконання резекції III сегмента з контролем гемостазу.

\subsubsection{Контроль гемостазу та жовчотоку}
- Перевірка гемостазу за допомогою біполярної коагуляції.

\subsection{Інтраопераційні ускладнення та їх корекція}
\begin{itemize}
    \item Кровотеча: контроль за допомогою гемостатичних засобів.
\end{itemize}

\subsection{Післяопераційне ведення пацієнта}
\begin{itemize}
    \item Контроль функції печінки.
    \item Рання мобілізація.
\end{itemize}

\subsection{Клінічний випадок}
\textbf{Пацієнт:} жінка, 60 років, пухлина Sg 3 розміром 4 см (аденома).

\textbf{Передопераційне обстеження:} КТ – утворення без ознак судинної інвазії.

\textbf{Операція:} лапароскопічна анатомічна резекція Sg 3, час 100 хв, крововтрата 80 мл.

\textbf{Результат:} без ускладнень, виписка на 3-й день.

\subsection{Висновки}
- Анатомічна резекція Sg 3 є ефективною процедурою для видалення пухлин.
- Вимагає високої точності хірургічної техніки.

\section{Лівобічна латеральна секцієектомія}
\subsection{Визначення та загальна характеристика}
Лівобічна латеральна секцієектомія – це хірургічна процедура, що передбачає видалення лівої латеральної частини печінки, включаючи сегменти II і III. Ця операція може бути проведена при наявності доброякісних або злоякісних утворень.

\subsection{Покази та протипокази}
\subsubsection{Покази:}
\begin{itemize}
    \item Гепатоцелюлярна карцинома (ГЦК) без судинної інвазії.
    \item Доброякісні пухлини (аденома, гемангіома).
\end{itemize}

\subsubsection{Протипокази:}
\begin{itemize}
    \item Дифузне ураження печінки.
    \item Глибоке розташування пухлини.
\end{itemize}

\subsection{Передопераційна підготовка}
\subsubsection{Оцінка пацієнта:}
\begin{itemize}
    \item Лабораторні тести: печінкові ферменти, білірубін.
    \item Візуалізація: КТ з контрастом.
\end{itemize}

\subsubsection{Планування втручання:}
\begin{itemize}
    \item Оцінка судинної анатомії.
    \item Використання 3D-моделювання.
\end{itemize}

\subsection{Технічне забезпечення операції}
\begin{itemize}
    \item Лапароскопічна стійка: 4K-візуалізація.
    \item Інструменти: CUSA, біполярна коагуляція.
\end{itemize}

\subsection{Оперативна техніка}
\subsubsection{Укладка пацієнта}
- Пацієнт у положенні на спині.

\subsubsection{Доступ та розташування троакарів}
- Камера – навколопупковий доступ.
- Робочі порти – у лівому підребер'ї.

\subsubsection{Мобілізація печінки}
- Часткова мобілізація лівої частини печінки.

\subsubsection{Техніка резекції}
- Виконання резекції лівобічної латеральної частини з контролем гемостазу.

\subsubsection{Контроль гемостазу та жовчотоку}
- Перевірка гемостазу за допомогою біполярної коагуляції.

\subsection{Інтраопераційні ускладнення та їх корекція}
\begin{itemize}
    \item Кровотеча: контроль за допомогою гемостатичних засобів.
\end{itemize}

\subsection{Післяопераційне ведення пацієнта}
\begin{itemize}
    \item Контроль функції печінки.
    \item Рання мобілізація.
\end{itemize}

\subsection{Клінічний випадок}
\textbf{Пацієнт:} жінка, 55 років, пухлина в лівій частині печінки розміром 5 см (аденома).

\textbf{Передопераційне обстеження:} КТ – утворення без ознак судинної інвазії.

\textbf{Операція:} лапароскопічна лівобічна латеральна секцієектомія, час 110 хв, крововтрата 70 мл.

\textbf{Результат:} без ускладнень, виписка на 4-й день.

\subsection{Висновки}
- Лівобічна латеральна секцієектомія є ефективною процедурою для видалення пухлин.
- Вимагає високої точності хірургічної техніки.

\section{Анатомічна резекція Sg 4}
\subsection{Визначення та загальна характеристика}
Анатомічна резекція Sg 4 – це операція, що передбачає видалення IV сегмента печінки, яка може бути виконана при наявності доброякісних або злоякісних утворень.

\subsection{Покази та протипокази}
\subsubsection{Покази:}
\begin{itemize}
    \item Гепатоцелюлярна карцинома (ГЦК) без судинної інвазії.
    \item Доброякісні пухлини (аденома, гемангіома).
\end{itemize}

\subsubsection{Протипокази:}
\begin{itemize}
    \item Дифузне ураження печінки.
    \item Глибоке розташування пухлини.
\end{itemize}

\subsection{Передопераційна підготовка}
\subsubsection{Оцінка пацієнта:}
\begin{itemize}
    \item Лабораторні тести: печінкові ферменти, білірубін.
    \item Візуалізація: КТ з контрастом.
\end{itemize}

\subsubsection{Планування втручання:}
\begin{itemize}
    \item Оцінка судинної анатомії.
    \item Використання 3D-моделювання.
\end{itemize}

\subsection{Технічне забезпечення операції}
\begin{itemize}
    \item Лапароскопічна стійка: 4K-візуалізація.
    \item Інструменти: CUSA, біполярна коагуляція.
\end{itemize}

\subsection{Оперативна техніка}
\subsubsection{Укладка пацієнта}
- Пацієнт у положенні на спині.

\subsubsection{Доступ та розташування троакарів}
- Камера – навколопупковий доступ.
- Робочі порти – у правому підребер'ї.

\subsubsection{Мобілізація печінки}
- Часткова мобілізація IV сегмента.

\subsubsection{Техніка резекції}
- Виконання резекції IV сегмента з контролем гемостазу.

\subsubsection{Контроль гемостазу та жовчотоку}
- Перевірка гемостазу за допомогою біполярної коагуляції.

\subsection{Інтраопераційні ускладнення та їх корекція}
\begin{itemize}
    \item Кровотеча: контроль за допомогою гемостатичних засобів.
\end{itemize}

\subsection{Післяопераційне ведення пацієнта}
\begin{itemize}
    \item Контроль функції печінки.
    \item Рання мобілізація.
\end{itemize}

\subsection{Клінічний випадок}
\textbf{Пацієнт:} чоловік, 62 роки, пухлина Sg 4 розміром 3 см (аденома).

\textbf{Передопераційне обстеження:} КТ – утворення без ознак судинної інвазії.

\textbf{Операція:} лапароскопічна анатомічна резекція Sg 4, час 95 хв, крововтрата 60 мл.

\textbf{Результат:} без ускладнень, виписка на 3-й день.

\subsection{Висновки}
- Анатомічна резекція Sg 4 є ефективною процедурою для видалення пухлин.
- Вимагає високої точності хірургічної техніки.

\section{Лівобічна гемігепатектомія}
\subsection{Визначення та загальна характеристика}
Лівобічна гемігепатектомія – це операція, що передбачає видалення лівої половини печінки, включаючи сегменти II, III та IV. Ця процедура може бути виконана при злоякісних або доброякісних утвореннях.

\subsection{Покази та протипокази}
\subsubsection{Покази:}
\begin{itemize}
    \item Гепатоцелюлярна карцинома (ГЦК) без судинної інвазії.
    \item Внутрішньопечінкова холангіокарцинома.
\end{itemize}

\subsubsection{Протипокази:}
\begin{itemize}
    \item Дифузне ураження печінки.
    \item Важка супутня патологія.
\end{itemize}

\subsection{Передопераційна підготовка}
\subsubsection{Оцінка пацієнта:}
\begin{itemize}
    \item Лабораторні тести: печінкові ферменти, білірубін.
    \item Візуалізація: КТ з контрастом.
\end{itemize}

\subsubsection{Планування втручання:}
\begin{itemize}
    \item Оцінка судинної анатомії.
    \item Використання 3D-моделювання.
\end{itemize}

\subsection{Технічне забезпечення операції}
\begin{itemize}
    \item Лапароскопічна стійка: 4K-візуалізація.
    \item Інструменти: CUSA, біполярна коагуляція.
\end{itemize}

\subsection{Оперативна техніка}
\subsubsection{Укладка пацієнта}
- Пацієнт у положенні на спині.

\subsubsection{Доступ та розташування троакарів}
- Камера – навколопупковий доступ.
- Робочі порти – у лівому підребер'ї.

\subsubsection{Мобілізація печінки}
- Часткова мобілізація лівої половини печінки.

\subsubsection{Техніка резекції}
- Виконання резекції лівобічної гемігепатектомії з контролем гемостазу.

\subsubsection{Контроль гемостазу та жовчотоку}
- Перевірка гемостазу за допомогою біполярної коагуляції.

\subsection{Інтраопераційні ускладнення та їх корекція}
\begin{itemize}
    \item Кровотеча: контроль за допомогою гемостатичних засобів.
\end{itemize}

\subsection{Післяопераційне ведення пацієнта}
\begin{itemize}
    \item Контроль функції печінки.
    \item Рання мобілізація.
\end{itemize}

\subsection{Клінічний випадок}
\textbf{Пацієнт:} жінка, 70 років, пухлина в лівій половині печінки розміром 6 см (аденома).

\textbf{Передопераційне обстеження:} КТ – утворення без ознак судинної інвазії.

\textbf{Операція:} лапароскопічна лівобічна гемігепатектомія, час 150 хв, крововтрата 120 мл.

\textbf{Результат:} без ускладнень, виписка на 5-й день.

\subsection{Висновки}
- Лівобічна гемігепатектомія є ефективною процедурою для видалення пухлин.
- Вимагає високої точності хірургічної техніки.

\section{Анатомічна резекція Sg 5}
\subsection{Визначення та загальна характеристика}
Анатомічна резекція Sg 5 – це операція, що передбачає видалення V сегмента печінки, яка може бути виконана при наявності доброякісних або злоякісних утворень.

\subsection{Покази та протипокази}
\subsubsection{Покази:}
\begin{itemize}
    \item Гепатоцелюлярна карцинома (ГЦК) без судинної інвазії.
    \item Доброякісні пухлини (аденома, гемангіома).
\end{itemize}

\subsubsection{Протипокази:}
\begin{itemize}
    \item Дифузне ураження печінки.
    \item Глибоке розташування пухлини.
\end{itemize}

\subsection{Передопераційна підготовка}
\subsubsection{Оцінка пацієнта:}
\begin{itemize}
    \item Лабораторні тести: печінкові ферменти, білірубін.
    \item Візуалізація: КТ з контрастом.
\end{itemize}

\subsubsection{Планування втручання:}
\begin{itemize}
    \item Оцінка судинної анатомії.
    \item Використання 3D-моделювання.
\end{itemize}

\subsection{Технічне забезпечення операції}
\begin{itemize}
    \item Лапароскопічна стійка: 4K-візуалізація.
    \item Інструменти: CUSA, біполярна коагуляція.
\end{itemize}

\subsection{Оперативна техніка}
\subsubsection{Укладка пацієнта}
- Пацієнт у положенні на спині.

\subsubsection{Доступ та розташування троакарів}
- Камера – навколопупковий доступ.
- Робочі порти – у правому підребер'ї.

\subsubsection{Мобілізація печінки}
- Часткова мобілізація V сегмента.

\subsubsection{Техніка резекції}
- Виконання резекції V сегмента з контролем гемостазу.

\subsubsection{Контроль гемостазу та жовчотоку}
- Перевірка гемостазу за допомогою біполярної коагуляції.

\subsection{Інтраопераційні ускладнення та їх корекція}
\begin{itemize}
    \item Кровотеча: контроль за допомогою гемостатичних засобів.
\end{itemize}

\subsection{Післяопераційне ведення пацієнта}
\begin{itemize}
    \item Контроль функції печінки.
    \item Рання мобілізація.
\end{itemize}

\subsection{Клінічний випадок}
\textbf{Пацієнт:} чоловік, 65 років, пухлина Sg 5 розміром 4 см (аденома).

\textbf{Передопераційне обстеження:} КТ – утворення без ознак судинної інвазії.

\textbf{Операція:} лапароскопічна анатомічна резекція Sg 5, час 100 хв, крововтрата 70 мл.

\textbf{Результат:} без ускладнень, виписка на 3-й день.

\subsection{Висновки}
- Анатомічна резекція Sg 5 є ефективною процедурою для видалення пухлин.
- Вимагає високої точності хірургічної техніки.

\section{Анатомічна резекція Sg 6}
\subsection{Визначення та загальна характеристика}
Анатомічна резекція Sg 6 – це операція, що передбачає видалення VI сегмента печінки, яка може бути виконана при наявності доброякісних або злоякісних утворень.

\subsection{Покази та протипокази}
\subsubsection{Покази:}
\begin{itemize}
    \item Гепатоцелюлярна карцинома (ГЦК) без судинної інвазії.
    \item Доброякісні пухлини (аденома, гемангіома).
\end{itemize}

\subsubsection{Протипокази:}
\begin{itemize}
    \item Дифузне ураження печінки.
    \item Глибоке розташування пухлини.
\end{itemize}

\subsection{Передопераційна підготовка}
\subsubsection{Оцінка пацієнта:}
\begin{itemize}
    \item Лабораторні тести: печінкові ферменти, білірубін.
    \item Візуалізація: КТ з контрастом.
\end{itemize}

\subsubsection{Планування втручання:}
\begin{itemize}
    \item Оцінка судинної анатомії.
    \item Використання 3D-моделювання.
\end{itemize}

\subsection{Технічне забезпечення операції}
\begin{itemize}
    \item Лапароскопічна стійка: 4K-візуалізація.
    \item Інструменти: CUSA, біполярна коагуляція.
\end{itemize}

\subsection{Оперативна техніка}
\subsubsection{Укладка пацієнта}
- Пацієнт у положенні на спині.

\subsubsection{Доступ та розташування троакарів}
- Камера – навколопупковий доступ.
- Робочі порти – у правому підребер'ї.

\subsubsection{Мобілізація печінки}
- Часткова мобілізація VI сегмента.

\subsubsection{Техніка резекції}
- Виконання резекції VI сегмента з контролем гемостазу.

\subsubsection{Контроль гемостазу та жовчотоку}
- Перевірка гемостазу за допомогою біполярної коагуляції.

\subsection{Інтраопераційні ускладнення та їх корекція}
\begin{itemize}
    \item Кровотеча: контроль за допомогою гемостатичних засобів.
\end{itemize}

\subsection{Післяопераційне ведення пацієнта}
\begin{itemize}
    \item Контроль функції печінки.
    \item Рання мобілізація.
\end{itemize}

\subsection{Клінічний випадок}
\textbf{Пацієнт:} жінка, 68 років, пухлина Sg 6 розміром 5 см (аденома).

\textbf{Передопераційне обстеження:} КТ – утворення без ознак судинної інвазії.

\textbf{Операція:} лапароскопічна анатомічна резекція Sg 6, час 120 хв, крововтрата 90 мл.

\textbf{Результат:} без ускладнень, виписка на 4-й день.

\subsection{Висновки}
- Анатомічна резекція Sg 6 є ефективною процедурою для видалення пухлин.
- Вимагає високої точності хірургічної техніки.

\section{Бісегментектомія 5-6}
\subsection{Визначення та загальна характеристика}
Бісегментектомія 5-6 – це операція, що передбачає видалення V та VI сегментів печінки, яка може бути виконана при наявності злоякісних або доброякісних утворень. Ця техніка дозволяє зберегти максимальну кількість здорової печінкової тканини.

\subsection{Покази та протипокази}
\subsubsection{Покази:}
\begin{itemize}
    \item Гепатоцелюлярна карцинома (ГЦК) без судинної інвазії.
    \item Метастази в V та VI сегментах.
\end{itemize}

\subsubsection{Протипокази:}
\begin{itemize}
    \item Дифузне ураження печінки.
    \item Глибоке розташування пухлини.
\end{itemize}

\subsection{Передопераційна підготовка}
\subsubsection{Оцінка пацієнта:}
\begin{itemize}
    \item Лабораторні тести: печінкові ферменти, білірубін.
    \item Візуалізація: КТ з контрастом.
\end{itemize}

\subsubsection{Планування втручання:}
\begin{itemize}
    \item Оцінка судинної анатомії.
    \item Використання 3D-моделювання.
\end{itemize}

\subsection{Технічне забезпечення операції}
\begin{itemize}
    \item Лапароскопічна стійка: 4K-візуалізація.
    \item Інструменти: CUSA, біполярна коагуляція.
\end{itemize}

\subsection{Оперативна техніка}
\subsubsection{Укладка пацієнта}
- Пацієнт у положенні на спині.

\subsubsection{Доступ та розташування троакарів}
- Камера – навколопупковий доступ.
- Робочі порти – у правому підребер'ї.

\subsubsection{Мобілізація печінки}
- Часткова мобілізація V та VI сегментів.

\subsubsection{Техніка резекції}
- Виконання бісегментектомії 5-6 з контролем гемостазу.

\subsubsection{Контроль гемостазу та жовчотоку}
- Перевірка гемостазу за допомогою біполярної коагуляції.

\subsection{Інтраопераційні ускладнення та їх корекція}
\begin{itemize}
    \item Кровотеча: контроль за допомогою гемостатичних засобів.
\end{itemize}

\subsection{Післяопераційне ведення пацієнта}
\begin{itemize}
    \item Контроль функції печінки.
    \item Рання мобілізація.
\end{itemize}

\subsection{Клінічний випадок}
\textbf{Пацієнт:} чоловік, 72 роки, пухлина в V сегменті розміром 6 см (аденома).

\textbf{Передопераційне обстеження:} КТ – утворення без ознак судинної інвазії.

\textbf{Операція:} лапароскопічна бісегментектомія 5-6, час 130 хв, крововтрата 100 мл.

\textbf{Результат:} без ускладнень, виписка на 4-й день.

\subsection{Висновки}
- Бісегментектомія 5-6 є ефективною процедурою для видалення пухлин.
- Вимагає високої точності хірургічної техніки.

\section{Правобічна гемігепатектомія}
\subsection{Визначення та загальна характеристика}
Правобічна гемігепатектомія (ЛапПГ) – це анатомічна резекція печінки, яка передбачає видалення IV, V, VI, VII та VIII сегментів. Дане втручання є технічно складним через необхідність мобілізації великого об’єму печінки, контроль печінкових вен та дисекцію ворітних структур.

ЛапПГ є ефективним методом лікування пацієнтів із первинними злоякісними пухлинами (ГЦК, ВПХК), метастатичними ураженнями, а також у рамках трансплантації печінки. Лапароскопічний підхід забезпечує швидше відновлення, меншу крововтрату та коротший період госпіталізації порівняно з відкритим втручанням.

\subsection{Покази та протипокази}
\subsubsection{Покази:}
\begin{itemize}
    \item Гепатоцелюлярна карцинома (ГЦК) у межах резектабельності (без судинної інвазії за межі печінки).
    \item Внутрішньопечінкова холангіокарцинома, локалізована в правій долі.
    \item Метастази колоректального раку в правій долі печінки.
    \item Доброякісні пухлини (аденома, фокальна нодулярна гіперплазія), що викликають симптоматику.
    \item Донорська правобічна гемігепатектомія.
\end{itemize}

\subsubsection{Протипокази:}
\begin{itemize}
    \item Тяжка супутня патологія (декомпенсований цироз, виражена портальна гіпертензія).
    \item Відсутність адекватного майбутнього печінкового залишку (FLR $<$30\% у пацієнтів без фіброзу, $<$40\% при цирозі).
    \item Дифузне ураження печінки або позапечінкове поширення пухлини.
\end{itemize}

\subsection{Передопераційна підготовка}
\subsubsection{Оцінка пацієнта:}
\begin{itemize}
    \item \textbf{Лабораторні тести:} рівень альбуміну, білірубіну, протромбінового часу.
    \item \textbf{Функціональні проби:} ICG-тест, оцінка FLR.
    \item \textbf{Онкологічна оцінка:} BCLC-стадіювання, КТ/МРТ з контрастом.
\end{itemize}

\subsubsection{Планування втручання:}
\begin{itemize}
    \item Визначення судинної анатомії за допомогою 3D-реконструкції.
    \item Оцінка можливості використання лапароскопічного Прінгла.
    \item Підготовка до можливих судинних реконструкцій.
\end{itemize}

\subsection{Технічне забезпечення операції}
\begin{itemize}
    \item \textbf{Лапароскопічна стійка:} 4K-візуалізація, ICG-флуоресценція.
    \item \textbf{Інструменти:} ендоскопічний ультразвук, CUSA, Thunderbeat, зшиваючі апарати.
    \item \textbf{Резекція судин:} судинні кліпси, Prolene 5-0 для судинної реконструкції (за потреби).
\end{itemize}

\section{Правобічна задня секцієектомія}
\subsection{Визначення та загальна характеристика}
Правобічна задня секцієектомія – це хірургічна процедура, що передбачає видалення задньої частини правої долі печінки, включаючи сегменти VI та VII. Ця операція може бути виконана при наявності доброякісних або злоякісних утворень.

\subsection{Покази та протипокази}
\subsubsection{Покази:}
\begin{itemize}
    \item Гепатоцелюлярна карцинома (ГЦК) без судинної інвазії.
    \item Метастази в правій задній частині печінки.
\end{itemize}

\subsubsection{Протипокази:}
\begin{itemize}
    \item Дифузне ураження печінки.
    \item Важка супутня патологія.
\end{itemize}

\subsection{Передопераційна підготовка}
\subsubsection{Оцінка пацієнта:}
\begin{itemize}
    \item Лабораторні тести: печінкові ферменти, білірубін.
    \item Візуалізація: КТ з контрастом.
\end{itemize}

\subsubsection{Планування втручання:}
\begin{itemize}
    \item Оцінка судинної анатомії.
    \item Використання 3D-моделювання.
\end{itemize}

\subsection{Технічне забезпечення операції}
\begin{itemize}
    \item Лапароскопічна стійка: 4K-візуалізація.
    \item Інструменти: CUSA, біполярна коагуляція.
\end{itemize}

\subsection{Оперативна техніка}
\subsubsection{Укладка пацієнта}
- Пацієнт у положенні на спині.

\subsubsection{Доступ та розташування троакарів}
- Камера – навколопупковий доступ.
- Робочі порти – у правому підребер'ї.

\subsubsection{Мобілізація печінки}
- Часткова мобілізація правої частини печінки.

\subsubsection{Техніка резекції}
- Виконання резекції правобічної задньої секцієектомії з контролем гемостазу.

\subsubsection{Контроль гемостазу та жовчотоку}
- Перевірка гемостазу за допомогою біполярної коагуляції.

\subsection{Інтраопераційні ускладнення та їх корекція}
\begin{itemize}
    \item Кровотеча: контроль за допомогою гемостатичних засобів.
\end{itemize}

\subsection{Післяопераційне ведення пацієнта}
\begin{itemize}
    \item Контроль функції печінки.
    \item Рання мобілізація.
\end{itemize}

\subsection{Клінічний випадок}
\textbf{Пацієнт:} жінка, 60 років, пухлина в правій задній частині печінки розміром 4 см (аденома).

\textbf{Передопераційне обстеження:} КТ – утворення без ознак судинної інвазії.

\textbf{Операція:} лапароскопічна правобічна задня секцієектомія, час 110 хв, крововтрата 80 мл.

\textbf{Результат:} без ускладнень, виписка на 4-й день.

\subsection{Висновки}
- Правобічна задня секцієектомія є ефективною процедурою для видалення пухлин.
- Вимагає високої точності хірургічної техніки.

\section{Анатомічна резекція Sg 7}
\subsection{Визначення та загальна характеристика}
Анатомічна резекція Sg 7 – це операція, що передбачає видалення VII сегмента печінки, яка може бути виконана при наявності доброякісних або злоякісних утворень.

\subsection{Покази та протипокази}
\subsubsection{Покази:}
\begin{itemize}
    \item Гепатоцелюлярна карцинома (ГЦК) без судинної інвазії.
    \item Доброякісні пухлини (аденома, гемангіома).
\end{itemize}

\subsubsection{Протипокази:}
\begin{itemize}
    \item Дифузне ураження печінки.
    \item Глибоке розташування пухлини.
\end{itemize}

\subsection{Передопераційна підготовка}
\subsubsection{Оцінка пацієнта:}
\begin{itemize}
    \item Лабораторні тести: печінкові ферменти, білірубін.
    \item Візуалізація: КТ з контрастом.
\end{itemize}

\subsubsection{Планування втручання:}
\begin{itemize}
    \item Оцінка судинної анатомії.
    \item Використання 3D-моделювання.
\end{itemize}

\subsection{Технічне забезпечення операції}
\begin{itemize}
    \item Лапароскопічна стійка: 4K-візуалізація.
    \item Інструменти: CUSA, біполярна коагуляція.
\end{itemize}

\subsection{Оперативна техніка}
\subsubsection{Укладка пацієнта}
- Пацієнт у положенні на спині.

\subsubsection{Доступ та розташування троакарів}
- Камера – навколопупковий доступ.
- Робочі порти – у правому підребер'ї.

\subsubsection{Мобілізація печінки}
- Часткова мобілізація VII сегмента.

\subsubsection{Техніка резекції}
- Виконання резекції VII сегмента з контролем гемостазу.

\subsubsection{Контроль гемостазу та жовчотоку}
- Перевірка гемостазу за допомогою біполярної коагуляції.

\subsection{Інтраопераційні ускладнення та їх корекція}
\begin{itemize}
    \item Кровотеча: контроль за допомогою гемостатичних засобів.
\end{itemize}

\subsection{Післяопераційне ведення пацієнта}
\begin{itemize}
    \item Контроль функції печінки.
    \item Рання мобілізація.
\end{itemize}

\subsection{Клінічний випадок}
\textbf{Пацієнт:} чоловік, 58 років, пухлина Sg 7 розміром 3 см (аденома).

\textbf{Передопераційне обстеження:} КТ – утворення без ознак судинної інвазії.

\textbf{Операція:} лапароскопічна анатомічна резекція Sg 7, час 90 хв, крововтрата 50 мл.

\textbf{Результат:} без ускладнень, виписка на 3-й день.

\subsection{Висновки}
- Анатомічна резекція Sg 7 є ефективною процедурою для видалення пухлин.
- Вимагає високої точності хірургічної техніки.

\section{Анатомічна резекція Sg 8}
\subsection{Визначення та загальна характеристика}
Анатомічна резекція Sg 8 – це операція, що передбачає видалення VIII сегмента печінки, яка може бути виконана при наявності доброякісних або злоякісних утворень.

\subsection{Покази та протипокази}
\subsubsection{Покази:}
\begin{itemize}
    \item Гепатоцелюлярна карцинома (ГЦК) без судинної інвазії.
    \item Доброякісні пухлини (аденома, гемангіома).
\end{itemize}

\subsubsection{Протипокази:}
\begin{itemize}
    \item Дифузне ураження печінки.
    \item Глибоке розташування пухлини.
\end{itemize}

\subsection{Передопераційна підготовка}
\subsubsection{Оцінка пацієнта:}
\begin{itemize}
    \item Лабораторні тести: печінкові ферменти, білірубін.
    \item Візуалізація: КТ з контрастом.
\end{itemize}

\subsubsection{Планування втручання:}
\begin{itemize}
    \item Оцінка судинної анатомії.
    \item Використання 3D-моделювання.
\end{itemize}

\subsection{Технічне забезпечення операції}
\begin{itemize}
    \item Лапароскопічна стійка: 4K-візуалізація.
    \item Інструменти: CUSA, біполярна коагуляція.
\end{itemize}

\subsection{Оперативна техніка}
\subsubsection{Укладка пацієнта}
- Пацієнт у положенні на спині.

\subsubsection{Доступ та розташування троакарів}
- Камера – навколопупковий доступ.
- Робочі порти – у правому підребер'ї.

\subsubsection{Мобілізація печінки}
- Часткова мобілізація VIII сегмента.

\subsubsection{Техніка резекції}
- Виконання резекції VIII сегмента з контролем гемостазу.

\subsubsection{Контроль гемостазу та жовчотоку}
- Перевірка гемостазу за допомогою біполярної коагуляції.

\subsection{Інтраопераційні ускладнення та їх корекція}
\begin{itemize}
    \item Кровотеча: контроль за допомогою гемостатичних засобів.
\end{itemize}

\subsection{Післяопераційне ведення пацієнта}
\begin{itemize}
    \item Контроль функції печінки.
    \item Рання мобілізація.
\end{itemize}

\subsection{Клінічний випадок}
\textbf{Пацієнт:} жінка, 65 років, пухлина Sg 8 розміром 5 см (аденома).

\textbf{Передопераційне обстеження:} КТ – утворення без ознак судинної інвазії.

\textbf{Операція:} лапароскопічна анатомічна резекція Sg 8, час 120 хв, крововтрата 90 мл.

\textbf{Результат:} без ускладнень, виписка на 4-й день.

\subsection{Висновки}
- Анатомічна резекція Sg 8 є ефективною процедурою для видалення пухлин.
- Вимагає високої точності хірургічної техніки.

\section{Мезогепатектомія}
\subsection{Визначення та загальна характеристика}
Мезогепатектомія – це хірургічна процедура, що передбачає видалення середньої частини печінки, включаючи сегменти IV, V та VI. Ця операція може бути виконана при наявності злоякісних або доброякісних утворень.

\subsection{Покази та протипокази}
\subsubsection{Покази:}
\begin{itemize}
    \item Гепатоцелюлярна карцинома (ГЦК) без судинної інвазії.
    \item Метастази в середній частині печінки.
\end{itemize}

\subsubsection{Протипокази:}
\begin{itemize}
    \item Дифузне ураження печінки.
    \item Важка супутня патологія.
\end{itemize}

\subsection{Передопераційна підготовка}
\subsubsection{Оцінка пацієнта:}
\begin{itemize}
    \item Лабораторні тести: печінкові ферменти, білірубін.
    \item Візуалізація: КТ з контрастом.
\end{itemize}

\subsubsection{Планування втручання:}
\begin{itemize}
    \item Оцінка судинної анатомії.
    \item Використання 3D-моделювання.
\end{itemize}

\subsection{Технічне забезпечення операції}
\begin{itemize}
    \item Лапароскопічна стійка: 4K-візуалізація.
    \item Інструменти: CUSA, біполярна коагуляція.
\end{itemize}

\subsection{Оперативна техніка}
\subsubsection{Укладка пацієнта}
- Пацієнт у положенні на спині.

\subsubsection{Доступ та розташування троакарів}
- Камера – навколопупковий доступ.
- Робочі порти – у правому підребер'ї.

\subsubsection{Мобілізація печінки}
- Часткова мобілізація середньої частини печінки.

\subsubsection{Техніка резекції}
- Виконання мезогепатектомії з контролем гемостазу.

\subsubsection{Контроль гемостазу та жовчотоку}
- Перевірка гемостазу за допомогою біполярної коагуляції.

\subsection{Інтраопераційні ускладнення та їх корекція}
\begin{itemize}
    \item Кровотеча: контроль за допомогою гемостатичних засобів.
\end{itemize}

\subsection{Післяопераційне ведення пацієнта}
\begin{itemize}
    \item Контроль функції печінки.
    \item Рання мобілізація.
\end{itemize}

\subsection{Клінічний випадок}
\textbf{Пацієнт:} чоловік, 75 років, пухлина в середній частині печінки розміром 7 см (аденома).

\textbf{Передопераційне обстеження:} КТ – утворення без ознак судинної інвазії.

\textbf{Операція:} лапароскопічна мезогепатектомія, час 140 хв, крововтрата 110 мл.

\textbf{Результат:} без ускладнень, виписка на 5-й день.

\subsection{Висновки}
- Мезогепатектомія є ефективною процедурою для видалення пухлин.
- Вимагає високої точності хірургічної техніки.

\printbibliography [heading=subbibliography]
\end{refsection}
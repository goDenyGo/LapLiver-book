\chapter[Планування та підготовка]{Планування та підготовка до лапароскопічних резекцій печінки}
\begin{refsection}

\section{Покази до виконання лапароскопічних резекцій печінки}

Технічні можливості лапароскопічної хірургії постійно зростають а хірургічна техніка вдосконалюється завдяки чому більшість втручаннь, що раніше виконувались лише у відкритому доступі зараз можлива і в лапароскопічному варіанті. Враховуючи сучасні досягнення покази до \acrshort{llr} практично не відрізняються від показів до \acrshort{olr} - лапароскопічний досуп не обмежує хірургічні можливості, проте додає певні властиві саме йому особливості. 

Серед показів до резекції печінки розрізняють злоякісну та доброякісну патологію. Серед онкопатології, на долю якої приходиться 60-80\% резекцій  найбільш частими показами до хірургічного лікування є первинні пухлини печінки та жовчних шляхів, а саме гепатоцелюлярна карцинома (\acrshort{hcc}), внутрішньопечінкова (або масформуюча) холангіокарцинома (\acrshort{ihcc}), перихіларна холангіокарцинома (\acrshort{phcc}), рак жовчного міхура (\acrshort{gbc}) та метастази колоректального раку в печінку \acrshort{crlm}. 

В цьому розділі ми зосередимось на тих видах хірургічної патології печінки, при яких застосування мініінвазивного підходу дає можливість отримати найкращі результати. 

\subsection{Гепатоцеллюларна карцинома}.

Гепатоцеллюлярна карцинома (\acrshort{hcc}) є п'ятою за частотою серед причин летальності від онкологічних захворюваннь. Причиною цього є  виявлення пізніх стадій захворювання через його асимптоматичний перебіг. Поширені форми \acrshort{hcc} характеризуються судинною інвазією, яка значно утруднює їх хірургічне лікування, а також наявністю у більшості хворих супутнього хронічного захворювання печінки та цирозу які суттєво погіршують печінкову функцію. Окрім того \acrshort{hcc} є хіміорезистентною пухлиною, що робить системну хіміотерапію не ефективною.

Сучасний підхід до лікування \acrshort{hcc} базується на виборі методу лікування в залежності від стадії пухлини. Найбільш часто вживаною системою стаціювання \acrshort{hcc} є барселонська (\acrfull{bclc}, \acrshort{bclc}) \cite{Llovet2003}. Згідно \acrshort{bclc} хірургічне лікування показано на ранній та дуже ранній стадіях \acrshort{hcc}, коли пухлина представлена солітарними резектабельними вузлами, а можливми опціями лікування є етанолова або радіочастотна абляція, відкрита або лапароскопічна резекція та трансплантація печінки. 

При дуже ранній стадії \acrshort{hcc} за \acrshort{bclc} з розміром вогнища до 2 см найбільш ефективні етанолова та радіочастотна абляція  \cite{Cucchetti2013}. Пацієнтам з \acrshort{hcc} в межах міланських критеріїв та декомпенсованим цирозом печінки оптимальним методом лікування є трансплантація печінки \cite{Colombo2016}. Для всіх інших пацієнтів з резектабельними формами \acrshort{hcc} методом першого вибору є резекція печінки \cite{Heimbach2018, Kudo2011}. 

\subsubsection{Анатомічна та неанатомічна резекція} 
\acrshort{hcc} --- агресивна високоінвазивна пухлина, яка має тенденцію до ураження малих та великих внутрішньопечінкових судин, периваскулярних компартментів та жовчних шляхів. Портальні тромби спричинені судинною інвазією можуть викликати кавернозну трансформацію та перивенозну колатеральну сітку. Мікросудинна інвазія це типова особливість \acrshort{hcc}, характерна переважно менш діфференційованим пухлинам, яка підтверджує агрессивну біологію пухлини. Незалежними предикторами мікроваскулярної інвазії є розмір пухлини більший 5 см., та менший ступінь дифференціації \cite{Zimmermann2017}. Інвазія в дрібні внутрішньопечінкові гілки ворітної вени може приводити до локального ретропортального кровотоку та периферійного внутрішньопечінкового розповсюдження у вигляді числених метастатичних вузлів по ходу судин в межах портального судинного басейну локальної анатомічної ділянки \cite{Kim2008}. 

Враховуючи схильність \acrshort{hcc} до локального метастазування в межах анатомічних ділянок методом вибору хірургічного лікування є анатомічна резекція печінки (\acrshort{alr}). На відміну від неанатомічної резекції печінки (\acrshort{non-alr}) \acrshort{alr} включає в себе ідентифікацію судинного бассейну анатомічної ділянки, що містить пухлину та видалення всієї її паренхіми. Онкологічні переваги \acrshort{alr} підтверждують більшість дослідженнь, так Makuuchi M. та співавтори \cite{Shindoh2016} в дослідженні 209 пацієнтів з цирозом класу А за Чайлдом та \acrshort{hcc} розміром $\leq$ 5 см, що були резектабельні як за допомогою \acrshort{alr} так і \acrshort{non-alr} показали перевагу \acrshort{alr} завдяки меншому ризику локальних рецидивів та більшій тривалості життя. Автори метааналізу \cite{Moris2018} 43 досліджень стверджують, що у 6839 пацієнтів яким була виконана \acrshort{alr} в порівнянні з 5590 пацієнтами з \acrshort{non-alr} були кращі загальна та безрецидивна виживаність та морбідність та рання післяопераційна летальність.

До переваг \acrshort{alr} відносять кращий онкологічний ефект, а до недоліків - вищий порівняно з \acrshort{non-alr} ризик післяопераційного порушення печінкової функції, пов'язаний з більшим об'ємом резекції, що особливо важливо для пацієнтів із цирозом печінки. Враховуючи це необхідний ретельний відбір пацієнтів для резекції печінки з приводу \acrshort{hcc}, що базується на оцінці рівню печінкової недостатності, ступеню портальної гіпертензії та загального статусу пацієнта. Американські клінічні настанови National Comprehensive Cancer Network (\acrshort{NCCN}) в якості кандидатів для резекції печінки пропонують пацієнтів з компенсованою печінковою функцією, солітарним вогнищем без макросудинної інвазії та майбутнім печінковим залишком (\acrshort{flr}) $\geq$ 20\% для здорової паренхіми та $\geq$ 30-40\% з адекватним кровопостачанням та жовчевідтоком. В клінічних настановах Європейської ассоціації вивчення хвороб печінки по лікуванню \acrshort{hcc} \cite{Galle2018a} для пацієнтів з цирозом печінки запропоновано спрощений алгорим визначення ризику \acrshort{alr}, що базується на об'ємі резекції, супеню портальної гіпертензії та печінкової недостатності. 

\subsubsection{Лапароскопічна та відкрита резекція печінки при \acrshort{hcc}} 
Окрім віддалених результатів на ефективність лікування впливає післяопераційна морбідність. Першим з двох основних факторів, які впливають на післяопераційні результати під час резекції печінки з приводу \acrshort{hcc} є крововтрата. Вищий об'єм крововтрати та замісної трансфузії препаратів крові асоційований з вищою з частотою післяопераційних ускладненнь та меншою віддаленою виживаністю \cite{DeBoer2007, Romano2012}. Крововтрата викликає імунодепрессію що сприяє розвитку хірургічних інфекцій, сепсису і збільшення ризику рецидиву в подальшому. Другим фактором є післяопераційна асцитопродукція у пацієнтів з цирозом печінки, яка є грізним ускладненням, що може призводити до великих втрат білка та рідини (до 5 літрів на добу), нагноєння післяопераційної рани та погіршання печінкової функції \cite{Ishii2014}. Окрім збільшення резистивності портального русла через його зменшення внаслідок резекції печінки та набряку печінкового залишку, механізм виникнення асцитопродукції пов'язують з декомпенсацією портальної гіпертензії внаслідок переривання портосистемної колатералізації під час лапаротомії \cite{Kanazawa2013}. 

Згідно існуючих даних, лапароскопічний підхід ефективно знижує ризики обох цих ускладненнь. За рахунок більш прецизійної техніки та позитивного тиску карбоксиперитонеуму \acrshort{llr} асоційовані з меншою інтраопераційною крововтратою, а збереження цілісності колатералей передньої черевної стінки дозволяє знизити частоту та інтенсивність післяопераційної асцитопродукції \cite{Truant2011}. Також \acrshort{llr} зменшують і загальну післяопераційну морбідніть у пацієнтів з \acrshort{hcc} на фоні цирозу при вираженій портальній гіпертензії та тромбоцитопенії < 100,000/мл про що свідчать результати міжнародного мультицентрового порівняльного псевдорандомізованого дослідження результатів лікування 1974 пацієнтів \cite{Ruzzenente2020}.

\emph{Враховуючи наведене вище оптимальними показами до \acrshort{llr} при \acrshort{hcc} є резектабельні форми пухлини всіх локалізацій без ураження магістральних судин або жовчних протоків на фоні здорової паренхіми або компенсованого цирозу печінки при відсутності показів до трансплантації печінки.}

\subsection{Рак біліарного тракту}

Рак біліарного тракту (\acrshort{btc}) - гетерогенна група захворюваннь до якої відносять внутрішньопечінкову (або масформуючу) холангіокарциному (\acrshort{ihcc}), перихіларну холангіокарциному (\acrshort{phcc}), рак жовчного міхура (\acrshort{gbc}) покази до лікування яких за допомогою \acrshort{llr} будуть розглянуті нижче, та рак дистального відділу холедоха і ампулярний рак, які не відносяться до теми цієї книги. 

\subsubsection{Внутрішньопечінкова масформуюча холангіокарцинома}

\acrshort{ihcc} є другою за частотою формою первинного рака печінки з високим потенціалом диссемінації та рецидивів, що походить із клітин проксимальних гілок жовчних протоків, на долю якої припадає до 40\% первинних пухлин печінки. Резекція печінки є єдиним потенційно радикальним методом та застосовується в якості першого етапу лікування \acrshort{ihcc}. Медіана виживаності пацієнтів після радикальної резекції складає 27 - 36 міс \cite{Buettner2017}. Не дивлячись на те, що вплив лімфаденектомії на виживаність при \acrshort{ihcc} остаточно не доказано рекомнедується її рутинне виконання з метою стадіювання процесу та вибору подальшого лікування. При досягненні R0 резекцийного краю необхідна системна адьювантна хіміотерапія, яка може бути доповнена хеморадіаційними методами в разі R1 резекції або позитивних лімфовузлів. При досягненні R2 резекцийного краю із макроскопічними залишками пухлини подальше лікування пацієнта необхідно проводити згідно рекомендацій до лікування нерезектабельних форм \acrshort{ihcc}. 

Резекція печінки при \acrshort{ihcc} є комплексною процедурою, технічна складність якої посилюється лімфаденектомією та можливою резекцією позапечінкових жовчних шляхів. Не дивлячись на те, що Саузґемптонські рекомендації не визначають однозначно місце \acrshort{llr} в лікуванні \acrshort{ihcc} через обмежену кількість накопиченого досвіду внаслідок низької частоти резектабельних форм, існує кілька порівняльних дослідженнь, що свідчать на користь лапароскопічного доступу. Так, група південнокорейських авторів на основі порівняння результатів 14 \acrshort{llr} з результатами 23 \acrshort{olr} у пацієнтів з \acrshort{ihcc} зазначає, що при порівняних онкологічних результатах пацієнти, що перенесли \acrshort{llr} мали меншу крововтрату та кращі ранні післяопераційні показники \cite{Lee2016a}. Характерною особливістю дослідження є те, що в нього включені лише невеликі пухлини, розміром $\leq$ 5 см. Ці данні також підтверджує двоцентрове псевдорандомізоване дослідження групи італійських та англійських вчених, в якому на основі вивчення результатів 208 \acrshort{llr} та \acrshort{olr} автори визначають \acrshort{llr} у відібраних пацієнтів  з \acrshort{ihcc}, як доступну та онкологічно ефективну процедуру, асоційовану з меншим ризиком післяопераційних ускладненнь \cite{Ratti2020}.


\subsubsection{Перихіларна холангіокарцинома}

На відміну від інших видів \acrshort{btc}, \acrshort{phcc}, це захворювання при якому показана можливість 10-річної виживаності після хірургічного лікування на рівні 14\% при адекватному відборі пацієнтв \cite{Juntermanns2019}. Через свої біологічні особливості \acrshort{phcc} має схильність до підслизового інфільтративного росту \cite{Sakamoto1998}, що обумовлює поширення пухлини за межі макроскопічно ураження, а також до інвазії магістральних судин ще до клінічної маніфестації у вигляді механічної жовтяниці \cite{Shimada2003}. Пухлина частіше розповсюджується лімфогенним та периневральним, ніж гематогенним шляхом \cite{Zimmermann2017}. Ці морфологічні особливості визначають агресивну хірургічну тактику та об'єм  втручання при \acrshort{phcc}, яке включає резекцію печінки відповідно рівню ураження із тотальною каудальною лобектомією, резекцію позапечінкових жовчних шляхів та розширену лімфаденектомію. За необхідості втручання може бути доповнене резекцією ворітної вени та печінкової артерії при їх пухлинній інвазії а також панкреатодуоденальною резекцією при інвазії дистального края холедоха \cite{Mizuno2019}. 

Через обширність та складність втручання в літературі доступні обмежені данні про мініінвазивне оперативне лікування \acrshort{phcc}. За данними огляду літератури та метааналізу групою авторів з нідерландів у 15 джерелах загалом виявлено 142 випадки виконання радикального мініінвазивного втручання з приводу \acrshort{phcc}, з яких 82 були \acrshort{llr} і 59 робот-асистованими резекціями \cite{Mizuno2019}. Дослідження виявило, що мініінвазивний підхід дає можливість отримати післяопераційну морбідність на рівні 24\% та летальність на рівні 3\%, що порівняно з результатами відкритих резекцій, та досягнути R0 резекції у 80\%. Не зважаючи на отримані оптимістичні результати, автори досить стримані у висновках, що обумовлено недостатньою кількістю накопиченого в світі досвіду та можливістю систематичної похибки через оцінку невеликих ретельно відібраних серій випадків. 

\subsubsection{Рак жовчного міхура}

\acrshort{gbc} є найбільш агресивною формою \acrshort{btc} з медіаною виживаності без лікування 6-10 місяців \cite{Lindner2018}, що пов'язано із асимптоматичним перебігом та пізнім виявленням. Радикальне хірургічна резекція може покращити результати виживаності із досягненням медіани 32 місяців. Сучасна тактика вибору об'єму хірургічного лікування \acrshort{gbc}, зазначена в рекомендаціях \acrshort{NCCN} з лікування гепатобіліарного раку, базується на стадії захворювання та обставинах його виявлення. При аксідентальному виявленні резектабельної ранньої форми \acrshort{gbc} під час холецистектомії остання доповнюється резекцією ложа жовчного міхура печінки. При виявленні пухлини під час планової проводки гістологічного препарату на стадії T1a (без інвазії підслизового шару) рекомендовано подальше спостереження, а на стадії T1b та вище, то в залежності від об'єму інвазії в паренхіму рекомендовано виконання другим етапом резекції Sg 4a-5 печінки або правобічної трисекцієектомії з лімфаденектомією. Резекція позапечінкових жовчних шляхів виконується в залежності від наявності ураження дистального краю міхурової протоки. Виконання резекції позапечінкових жовчних шляхів з метою розширення об'єму лімфаденектомії не є виправданим.  При доопераційному виявленні \acrshort{gbc} план втручання обирається в залежності від об'єму ураження.

В рекомендаціях по лікуванню \acrshort{btc} японської спілки гепатобіліарних хірургів 2015 року зазначається, що \acrshort{gbc} є протипоказом до лапароскопічної холецистектомії у зв'язку із високим ризиком диссемінації та портових метастазів \cite{Miyazaki2015}. Проте в більш пізньому міжнародному експертному консенсусі з лапароскопічної хірургії \acrshort{gbc} 2019 року наголошується, що лапароскопічний доступ не асоційований із погіршенням виживаності при ранніх стадіях \acrshort{gbc} (Т1,Т2) не ассоційованого із гострим холециститом, якщо виконується радикальна резекція, а доведеною причиною диссемінації є пошкодження цілісності стінки жовчного міхура та видалення макропрепарату без використання контейнера. У висновках консенсуса зазначається, що незважаючи на те, що \acrshort{llr} при \acrshort{gbc} є втручанням на ранніх стадіях вивчення, вона не погіршує прогноз, а у ретельно відібраних пацієнтів може покращувати результат операції. 


\emph{Таким чином, на теперішній момент \acrshort{llr} при всіх видах \acrshort{btc} є новою перспективною методикою на етапі вивчення, тому виконання цих операцій обмежено спеціалізованими центрами гепатобіліарної хірургії з великим досвідом мініінвазивних втручаннь.}

\subsection{Метастатичні ураження печінки}

\subsubsection{Метастатичний колоректальний рак}

\subsubsection{Метастатичний рак печінки не колоректального походження}


\subsection{Доброякісна патологія печінки}

\subsubsection{Аденоми печінки}

\subsubsection{Кистозні ураження печінки}

\subsubsection{Внутрішньопечінковий холелітіаз}


%%%%%%%%%%%%%%%%%%%%%%%%%%%%%%%%%%%%%%%%%%%%%%%%%%%%%%%%
%%%%%%%%%%%%%%%%%%%%%%%%%%%%%%%%%%%%%%%%%%%%%%%%%%%%%%%%
%%%%%%%%%%%%%%%%%%%%%%%%%%%%%%%%%%%%%%%%%%%%%%%%%%%%%%%%




\section{Лапароскопічний погляд на хірургічну анатомію печінки}

\subsection{Сучасні принципи хірургічної анатомії печінки}

\subsection{Портальна сегментація печінки}

\subsection{Капсула Лаенека та «вхідні ворота»}

\subsection{Печінкові вени як направляючі для плошини резекції}

\subsection{Анатомія першого сегменту печінки}


\section{Технічне забезпечення лапароскопічних резекцій печінки}

\subsection{Вимоги до лапароскопічної стійки}

\subsection{Інструменти доступу та видалення препарату}

\subsection{Базові інструменти}

\subsection{Зшиваючі аппарати}

\subsection{Хірургічні енергії}

\subsection{Спеціалізоване обладнання}


\printbibliography [heading=subbibliography]
\end{refsection}
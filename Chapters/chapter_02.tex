\chapter[Планування та підготовка]{Планування та підготовка до лапароскопічних резекцій печінки}
\begin{refsection}
\section{Покази до виконання лапароскопічних резекцій печінки}

\subsection{Гепатоцеллюларна карцинома}

\subsection{Метастатичний колоректальний рак}

\subsection{Рак жовчного міхура}

\subsection{Периферійна масформуюча холангіокарцинома}

\subsection{Хіларна холангіокарцинома}

\subsection{Доброякісні новоутворення печінки}

\section{Лапароскопічний погляд на хірургічну анатомію печінки}

\subsection{Сучасні принципи хірургічної анатомії печінки}

\subsection{Портальна сегментація печінки}

\subsection{Капсула Лаенека та «вхідні ворота»}

\subsection{Печінкові вени як направляючі для плошини резекції}

\subsection{Анатомія першого сегменту печінки}

\section{Технічне забезпечення лапароскопічних резекцій печінки}

\subsection{Вимоги до лапароскопічної стійки}

\subsection{Інструменти доступу та видалення препарату}

\subsection{Базові інструменти}

\subsection{Зшиваючі аппарати}

\subsection{Хірургічні енергії}

\subsection{Спеціалізоване обладнання}

\printbibliography [heading=subbibliography]
\end{refsection}
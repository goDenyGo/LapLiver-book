\chapter{Практичні аспекти лапароскопічних резекцій печінки}
\begin{refsection}

\section{Технічне забезпечення лапароскопічних резекцій печінки}

\subsection{Вимоги до лапароскопічної стійки}

\subsection{Інструменти доступу та видалення препарату}

\subsection{Базові інструменти}

\subsection{Зшиваючі аппарати}

\subsection{Хірургічні енергії}

\subsection{Спеціалізоване обладнання}

\section{Етапи лапароскопічних резекцій печінки}


\subsection{Укладка пацієнта}

\subsubsection{Французька позиція}

\subsubsection{Японська позиція}

\subsubsection{Інші види позицій}

\subsection{Доступ та розстановка троакарів}

\subsubsection{Методики постановки першого троакару}

\subsubsection{Позиціонування троакарів}

\subsubsection{Трансплевральний доступ}

\subsubsection{Торакальний трансдіафрагмальний доступ}

\subsection{Мобілізація печінки}

\subsubsection{Мобілізація кавальних воріт}

\subsubsection{Мобілізація лівої трикутної зв'язки}

\subsubsection{Мобілізація правої трикутної зв'язки}

\subsubsection{Мобілізація запечінкового сегменту нижньої порожнистої вени}

\subsection{Прийом Прінгла}

\subsubsection{Покази та правила виконання}

\subsubsection{Екстракорпоральний варіант}

\subsubsection{Інтракорпоральний варіант}

\subsection{Обробка портальних структур}

\subsubsection{Глісоновий підхід}

\subsubsection{Індивідуальне лігування}

\subsection{Транссекція паренхіми та методи гемостазу}

\subsubsection{Методи транссекції паренхіми та їх порівняння}

\subsubsection{Краш-клампінг}

\subsubsection{Транссекція за допомогою ультразвукового диссектора-аспіратора}

\subsubsection{Обробка внутрішньопаренхіматозних структур}

\subsection{Евакуація препарата}

\subsection{Типові помилки, їх профілактика та вирішення}

\subsection{Ускладнення після \acrshort{llr} та їх коррекція та профілактика}

\subsubsection{Інтраопераційні ускладнення}

\subsubsection{Післяопераційні ускладнення}

\printbibliography [heading=subbibliography]
\end{refsection}
\newpage

\noindent
\begin{minipage}{0.65\textwidth}
Сучасна хірургія печінки зазнала значних змін завдяки розвитку мінімально інвазивних технологій. Історично, лапароскопічна хірургія розвивалася протягом останніх кількох десятиліть, починаючи з перших експериментальних операцій, які часто супроводжувалися високими ризиками та невизначеністю. Проте, завдяки наполегливій праці хірургів та інженерів, ця методика стала стандартом у провідних клініках світу. Лапароскопічні резекції печінки, які ще двадцять років тому здавалися технічно неможливими, сьогодні вважаються золотим стандартом лікування. Ця трансформація не лише зменшила травматичність операцій, але й значно покращила післяопераційне відновлення пацієнтів, що є критично важливим аспектом у хірургії. Важливість навчання та симуляційних технологій у підготовці нової генерації хірургів не можна переоцінити; ці інструменти дозволяють лікарям отримувати необхідні навички в безпечному середовищі. Автори цієї книги не лише детально описують технічні аспекти операцій, а й розкривають клінічні випадки, що робить книгу надзвичайно цінною для практикуючих хірургів та молодих фахівців. Це видання є унікальним внеском у навчання та практику хірургів, які прагнуть опанувати цей складний, але ефективний метод, надаючи їм необхідні знання для успішного виконання лапароскопічних втручань.  
\vspace{30pt} 
\vfill   
\hfill \textit{Дуже Поважний Професор}
\vfill  
\end{minipage}
\hfill
\begin{minipage}{0.3\textwidth}
    \centering
    \includegraphics[width=\textwidth]{Illustrations/Preface/image1.png}
\end{minipage}

\newpage

\noindent
\begin{minipage}{0.65\textwidth}
Видання, яке ви тримаєте в руках, є результатом багаторічного досвіду та глибокого аналізу розвитку лапароскопічної хірургії печінки. У цій книзі автори не тільки узагальнили світовий досвід, а й представили власні досягнення та інноваційні підходи. Порівняння відкритих та лапароскопічних резекцій печінки виявляє численні переваги останніх, включаючи зменшення післяопераційних ускладнень та скорочення термінів госпіталізації. Однак, існують і певні обмеження, які потрібно враховувати. Мій власний досвід впровадження лапароскопічних технологій у хірургічну практику показав, що основними викликами були не лише технічні аспекти, але й необхідність навчання команди та адаптації до нових методик. Сучасні технології, такі як 3D-візуалізація, ICG-навідна резекція та роботизована хірургія, відіграють важливу роль у покращенні безпеки втручань, дозволяючи лікарям виконувати складні операції з максимальною точністю. Ця книга стане незамінним посібником для тих, хто прагне вдосконалити свої навички у сфері сучасної гепатобіліарної хірургії, сприяючи поширенню лапароскопічних резекцій печінки серед хірургів та підвищенню якості лікування пацієнтів.  
\\*
\vspace{30pt} 
\vfill 
\hfill \textit{Ще Один Дуже Поважний Професор}
\vfill 
\end{minipage}
\hfill
\begin{minipage}{0.3\textwidth}
    \centering
    \includegraphics[width=\textwidth]{Illustrations/Preface/image2.png}
\end{minipage}

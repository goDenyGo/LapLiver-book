\begin{refsection}
\chapter{Загальна частина}
\section{Історія розвитку лапароскопічних резекій печінки}
\subsection{Перший досвід}
За минулі три десятирічча мініінвазивні втручання радикально змінили підхід до лікування хворих хірургічного та онкологічного профілю  завдяки значному поліпшенню ранніх та віддалених результатів в урології, гінекології, абдомінальній та торокальній хірургії. Такий прогрес став можливим завдяки інтенсивному розвитку медичних технологій які змінили сам принцип хірургчного втручання. Системи візуалізації вдосконалили погляд хірурга додавши йому збільшення та змінний кут напрямку обзору, а лапароскопічні інструменти додали точність рухам в умовах обмеженого простору зробивши можливим виконання навіть надскладних втручаннь втручаннь на органах гепатобіліарної зони. 
В своєму розвитку \acrfull{llr} пройшла шлях від експериментальної методики до стандарту надання допомоги при пухлинах певних локалізацій\cite{AbuHilal2017a}. Перша згадка про виконання резекції печінки в лапароскопічному доступі належить Reich H. \cite{Reich1991a} та датована 1991 роком. Майже одночасно стали з'являтись  

\subsection{Погоджувальні конференції}
\section{Сучасні технічні можливості лапароскопічної резекційної хірургії печінки}
\subsection{Обширні анатомічні резекції печінки}
\subsection{Судинні реконструкції}
\subsection{Біліарні реконструкції}
\subsection{Розширена лімфодиссекція}
\subsection{ALPPS}
\printbibliography[heading=subbibliography]
\end{refsection}



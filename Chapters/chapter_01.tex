\begin{refsection}
\chapter{Загальна частина}
\section{Історія розвитку лапароскопічних резекій печінки}
\subsection{Перший досвід}
Мініінвазивні втручання радикально змінили хірургічну практику за останні три десятиріччя та призвели до суттєвого покращення результатів за рахунок зменшення частоти післяопераційних ускладненнь, тривалості госпіталізації та співвідношення вартості до еффективності лікування в різних хірургічних спеціальностях, включаючи колоректальну хірургію, урологію, гінекологію та торокальну хірургію. Природньо, що зацікавленість хірургів в лапароскопічному доступі швидко розповсюдилась на гепатобіліарні втручання.

Радикальна резекція печінки є одним з найбільш ефективних методів лікування для більшості новоутворень печінки. Спроби виконання гепатобіліарних втручаннь в лапароскопічному варіанті було розпочато в 1987 році із першої лапароскопічної холецистектомії \cite{Litynski}. Проте технічні обмеження та недоліки обладнання призвели до того, що перша резекція печінки в лапароскопічному варіанті була виконана лише чотири роки потому Reich H. \cite{Reich1991a} та невдовзі незалежно Katkhouda N. \cite{Katkhouda1992} та Gagner M. \cite{GAGNER1992}. Автори перших публікацій дійшли висновку, що результати лапароскопічних операцій порівняні із традиційними відкритими втручаннями а лапароскопічна резекція печінки (\acrshort{llr}) є безпечним та ефективним методом лікування. Відтоді \acrshort{llr} стала потенційною альтернативою відкритій резекції печінки (\acrshort{olr}), проте розвиток методики був достатньо повільним, і наступні декілька років публікації містять лише опис поодиноких випадків атипових резекцій \cite{Cunningham1995, Klotz1993}. 

Перша анатомічна лапароскопічна лівобічна латеральна секцієектомія(\acrshort{llls}) виконана в 1996 році відновила інтерес до \acrshort{llr} не дивлячись на те, що перші випадки були конвертовані у відкриті втручання через масивну інтраопераційну кровотечу \cite{Hashizume1995, Azagra1996}.  Зі збільшенням досвіду кількість \acrshort{llls} прогресивно збільшувалась, що  дозволило розширити можливості \acrshort{llr} до гемігепатектомії, складних сегментектомій, трисекцієектомії та навіть донорських резекцій при трансплантації печінки від живого донора \cite{Dagher2009, Cherqui2002, Jia2018}.
\subsection{Погоджувальні конференції}

Аккумуляція першого досвіду виконання \acrshort{llr} поставила перед хірургічною спільнотою завдання відповіді на два ключових питання: по-перше це безпека та відтворюваність методики а по-друге її онкологічна ефективність. Для відповіді на ці питання та створення клінічних рекомендацій стосовно застосування \acrshort{llr} було послідовно проведено декілька погоджувальних конференцій, висновки яких відображуть процес становлення лапароскопічного методу та зміну відношення до нього хірургічного загалу.

\subsubsection{Перша погоджувальна конференція} 
Перша погоджувальна конференція відбулась в Луізвілі в листопаді 2008 року за участі 45 міжнародних експертів з гепатобіліарної хірургії, що спеціалізувались як на лапароскопічних так і на відкритих втручаннях \cite{Buell2009}. За результатами обговорення до  \acrshort{llr} були віднесені чисто лапароскопічні, хенд-асистовані резекції печінки (\acrshort{hals}) та гибридні резекції печінки (\acrshort{hlr}). Прийнятними для видалення за допомогою  \acrshort{llr} були визначені утворення, розміром менше 5 см, розташовані в 2 - 6 сегментах печінки, а \acrshort{llls} було рекомендовано розглядати в якості стандартної практики. Також була показана технічна можливість видалення новоутворень, локалізованих в будь яких сегментах печінки в лапароскопічному варіанті, проте обширні резекції печінки було рекомендовано зарезервувати за спеціалізованими гепатобіліарними центрами з великим досвідом як лапароскопічних втручаннь, так і резекційної хірургії печінки. 

Конверсію рекомендовано розглядати скоріше як необхідний крок для безпечного завершення складного втручання, ніж як ускладнення. Перед ургентною конверсією з приводу кровотечі хірург повинен докласти максимальних зусиль до зупинки кровотечі лапароскопічними методами. 

Вперше була показана можливість виконання донорського забору печінки при трансплантації від живого донора у дітей в лапароскопічному варіанті. Донорську \acrshort{llls} було оцінено, як експериментальну методику, що має великий потенціал але потребує детального вивчення. 

Вцілому консенсус визначив \acrshort{llr} як безпечну альтернативу традиційним резекціям та рекомендував методику до подальшого вивчення та більш широкого впровадження досвідченими гепатобіліарними хірургами.

\subsubsection{Друга погоджувальна конференція} 
На відміну від першої, друга погоджувальна конференція, яка відбулась в японському місті Моріока в жовтні 2014 році \cite{Kaneko2015}, була побудована за Цюріхсько-Датською моделлю, та включала в себе експертну панель з 43 досвідчених хірургів та 9 членів жюрі з 18 країн які оцінювали результати \acrshort{llr}. Для оцінки були запропоновані 17 запитаннь в категоріях переваги, ризики та технічні аспекти \acrshort{llr}, відповідаючи на які жюрі сформувало рекомендації. Доказова якість рекомендацій була оцінена за шкалою GRADE а ступінь розробки втручаннь за системою IDEAL \cite{Guyatt2008, McCulloch2009}. Для об'єму резекції було запропоноване класичне визначення: до малих резекцій (Minor) було віднести резекції 2 та меньше сегментів а до великих (Major) 3 та більше сегментів. Але зважаючи на те, що технічна складність \acrshort{llls} та правобічної задньої або передньої секцієектомії в лапароскопічному варіанті значно відрізняються, резекції, до складу яких входять сегменти 7 або 8 було віднесено до великих. 

За висновками експертного жюрі як малі так і великі \acrshort{llr} не гірші за відкриті в показниках операційної летальності, післяопераційних ускладненнь, чистоті резекційного краю, загальної виживаності та вартості операції та мають перевагу в більш короткому терміні перебування в стаціонарі, меншій крововтраті. Експерти погодились з тим, що результати лапароскопічних донорських заборів не відрізнялись від відкритих у високоспеціалізованих центрах. У якості додаткового висновку жюрі заключило, що великі \acrshort{llr} вимагають високого рівню хірургічних навичок та тривалої кривої навчання.
Основним досягненням конференції було широке визнання \acrshort{llr} за більшістю показників є не гіршими, а за окремими показниками кращими за відкриті втручання та рекомендацію використання малих \acrshort{llr} як визнаного стандарту практики.

\subsubsection{Третя погоджувальна конференція та створення клінічних рекомендацій} 
Експотенціальний ріст кількості \acrshort{llr} призвів до того, що в лютому 2017 року в Саузхемптоні була проведена третя погоджувальна конференція \cite{AbuHilal2017a}, яка мала на меті створення загальноєвропейських клінічних рекомендацій. В процесі підготовки було залучено експертну панель з 11 досвідчених гепатобіліарних хірургів, частина з яких мала досвід лише відкритих резекцій печінки а частина як відкритих так і \acrshort{llr}. Після огляду аналізу 647 джерел, відібраних за допомогою критеріїв включення експертами були сформовані рекомендації в п'яти ключових напрямках: покази, відбір пацієнтів, види втручаннь, технічні особливості та імплементація.

Згідно з рекомендаціями \acrshort{llr} показані для лікування метахронних колоректальних метастазів та гепатоцеллюлярної карциноми так як ассоційовані зі зниженням крововтрати, післяопераційного асциту, печінкової недостатності та терміну перебування в стаціонарі порівняно із відкритими втручаннями при порівняній тривалості операції, частоті R0 краю резекції та рівні рецидивів. Також \acrshort{llr} показані для лікування доброякісної вогнищевої патології завдяки суттєвому зниженню післяопераційних ускладненнь, больового синдрому та терміну перебування в стаціонарі, що підтверджено на великих серіях пацієнтів, у тому числі із великими резекціями. Донорські гепатектомії наразі не є добре стандартизованими процедурами та зарезервовані за високоспеціалізованими центрами.

Що до відбору пацієнтів, то \acrshort{llr} добре показали себе у хворих з вираженою коморбідністю та можуть бути рекомендовані для пацієнтів з ожирінням та пацієнтів старшого віку. Є данні, що свідчать про полегшення перебігу повторних (відкритих або лапароскопічних) резекцій печінки у пацієнтів, що перенесли \acrshort{llr} в якості первинної операції. Складні випадки з великими новоутвореннями (> 10 см) та близкістю до магістральних судин не є протипоказами до \acrshort{llr}, так як можуть бути виконані з аналогічною відкритим операціям  морбідністю.

Усі види \acrshort{llr}, як великі так і малі, асоційовані зі зменшенням інтраопераційної крововтрати, післяопераційних ускладненнь та терміну перебування в стаціонарі та аналогічними показниками онкологічної результативності порівняно з віткритими втручаннями. Великі втручання та втручання на задніх сегментах пов'язані із більшою складністю та тривалістю операції, проте в експертних центрах можуть бути досягнуті периопераційні результати аналогічні малим \acrshort{llr}.

Жодна з існуючих технік виконання \acrshort{llr} (\acrshort{hals}, \acrshort{hlr} або чисто лапароскопічна) не показала повної переваги над іншими, проте вважається, що \acrshort{hals} та \acrshort{hlr} є перехідними до чисто лапароскопічної техніки, те ж саме стосується техніки транссекції паренхіми - краш-кламп, використання CUSA або інших хірургічних енергій. Для диссекції портальних структур більшисть хірургів використовують ізольоване лігування, проте глісоновий підхід показав аналогічні результати. Для контролю кровотечі під час \acrshort{llr} рекомендовано використовувати лапароскопічний прийом Прінгла та анастезію з низьким центральним венозним тиском (\acrshort{cvp}). Факторами ризику конверсії на відкрите втручання є високий індекс маси тіла, розмір пухлини, локалізація ураження в постеролатеральних сегментах та цирроз. Перед виконанням ургентної конверсії рекомендовано досягти тимчасового гемостаза лапароскопічними методами.

Крива навчання малих \acrshort{llr} складає 60 випадків для хірурга, що має досвід відкритих резекцій печінки. Для великих \acrshort{llr} цей показник становить 55 операцій, при умові успішного проходження кривої для малих \acrshort{llr}. Впровадження \acrshort{llr} не повинно відбуватись в ізоляції від відкритої хірургії печінки. Для кожного спелізованого гепатобіліарного центру рекомендована наявність не менше двох хірургів, спеціалізованих на \acrshort{llr}.

Таким чином, якщо послідовно проаналізувати висновки всіх погоджувальних конференцій стає зрозумілим, що методика \acrshort{llr} успішно пройшла етапи розробки, первинної оцінки та широкого впровадження базуючись на принципах доказової медицини. Чисельна кількість дослідженнь на великих групах пацієнтів \cite{Ciria2016b, Takahara2016, Berardi2017}, в тому числі два рандомізоаних клінічних дослідження \cite{Fretland2018b, Robles-Campos2019} свідчать про перевагу \acrshort{llr} над традиційними відкритими втручаннями в періопераційних показниках зі збереженням онкологічної ефективності.

\section{Сучасні технічні можливості лапароскопічної резекційної хірургії печінки}


\subsection{Обширні анатомічні резекції печінки}
\subsection{Судинні реконструкції}
\subsection{Біліарні реконструкції}
\subsection{Розширена лімфодиссекція}
\subsection{ALPPS}
\printbibliography[heading=subbibliography]
\end{refsection}



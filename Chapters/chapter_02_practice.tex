\chapter{Практичні аспекти лапароскопічних резекцій печінки}
\begin{refsection}

\section{Обстеження та передопераційна підготовка пацієнтів}

\subsection{Передопераційне обстеження}

Передопераційне обстеження пацієнтів з попередньо встановленим діагнозом, що плануються на \acrshort{llr} має вирішувати дві основіні клінічні задачі: 1) планування об'єму втручання та 2) виявлення можливих протипоказів до операції та супутніх захворюваннь. 

До дослідженнь, які вирішують задачу верифікації діагнозу та планування операції відносять:
\begin{itemize}
  
  \item \textbf{Контраст-підсилена спіральна комп'ютерна томографія (\acrshort{ct}) черевної порожнини та грудної клітини} -- основний метод візуалізації новоутвореннь печінки, який дозволяє оцінити локалізацію ураження, взаємовідношення його до магістральних внутрішньопечінкових структур і анатомічні особливості кровопостачання органа. За допомогою \acrshort{ct}-волюметрії оцінюють розміри потенційного печінкового залишку, що дозволяє обрати об'єм резекції. 
  
  \item \textbf{Контраст-підсилена магнітно-резонансна томографія (\acrshort{mri}) печінки з режимом зваженої дифузії та холангіографією} -- ефективний метод дифференційної діагностики, що дозволяє ідентифікувати вогнища невеликого розміру. \acrshort{mri}-холангіографія є неінвазивним способом отриманя інформації про взаємовідношення новоутворення печінки до жовчних шляхів. Найчастіше \acrshort{mri} використовується в комбінації з \acrshort{ct} так як методи доповнюють один інший. 
  
  \item \textbf{Ультразвукове дослідження (\acrshort{us}) черевної порожнини} використовують як допоміжний метод візуалізації а також функціональної оцінки вісцерального кровотоку в Допплер-режимі. Інтраопераційний (\acrshort{us}) дозволяє візуалізувати новоутворення розташовані глибоко в паренхімі.
  
  \item \textbf{Лабораторні тести.} Інтегральним маркером  стану печінки є розгорнутий біохімічний аналіз крові. Оцінити його можна за допомогою шкал MELD та Child-Pugh. Стадіювання за цими системами дає можливість оцінити ризики післяопераційної печінкової недостатності у пацієнтів що мають супутнє дифузне захворювання печінки (наприклад цироз або синдром обструкції синусоїдів).
  
  \item \textbf{Функціональна проба з індоціаніном зеленим \acrshort{icg}}. \acrshort{icg} є речовиною що виводиться з організму виключно через жовч. Визначення концентрації \acrshort{icg} на 15 хвилині після введення дозволяє оцінити екскреторну функцію печінки. Наявність більше ніж 10\% від початкової концентрації говорить про порушення функції печінки.
  
  \item \textbf{Вимірювання градієнту тиску заклинювання печінкових вен (для пацієнтів з портальною гіпертензією)} є інвазивним ендоваскулярним методом прямої оцінки портального тиску. При цьому дослідженні манометром під'єднаним до  ангіохірургічного катетера вимірюється різниця тисків між вільним положенням в \acrshort{ivc} та заклиненим положенням в гілку печінкової вени. Підвищення цієї різниці тисків вище 10 мм.рт.ст свідчить про наявність вираженої портальної гіпертензії.

\end{itemize}

Для виявлення супутньої патології та можливих протипоказів виконують:

\begin{itemize}
  \item розширені загальний та біохімічний аналізи крові, коагулограмма, загальний аналіз сечі
  \item гастро- та колоноскопія
  \item УЗД черевної порожнини
  \item Електрокардіографія, Ехокардіографія
  \item Консультації суміжних спеціалістів за наявності показів (кардіолог, гінеколог/уролог, судинний хірург та ін.) 
\end{itemize}

Наведений перелік обстеженнь є базовим, та за необхідності може бути розширений додатковими дослідженнями. 

\subsection{Відомі протипокази}

Протипокази до \acrshort{llr} можна умовно поділити на абсолютні та відносні. 

До \textbf{абсолютних} протипоказів відносять явища нестабільної гемодинаміки (обумовлені наприклад септичним станом пацієнта) та відсутність толерантності до карбоксиперитонеуму внаслідок серцево-судинної недостатності. Такі стани є загрозливими для життя та як правило потребують попередньої коррекції. При наявності життєвих показів до оперативного лікування у таких хворих перевага повинна надаватись відкритим втручанням. 

\textbf{Відносними} протипоказами до виконання резекції печінки в лапароскопічному варіанті є: 
\begin{itemize}
  \item \textit{Стан після перенесеного перитоніту.} Сама по собі наявність попереднього втручання не є протипоказом до \acrshort{llr}, так як в більшості випадків лапароскопічних засобів достатньо для вісцеролізу. Проте виражені запальні процеси в черевній порожнині можуть призводити до формування щільних злук по типу "панциря", що значно утруднює операцію.
  
  \item \textit{Наявність великої пухлини або гепатомегалія.} Це потребує прикладання до органу більшої фізичної сили, що може призводити до їх розривів та кровотеч. Критичним для прийняття рішення є не абсолютний розмір пухлини, а його відношення до об'єму вільної черевної порожнини.
  
  \item \textit{Інвазія пухлини в оточуючі органи, яка потребує їх резекції.} Наявність процесу такої розповсюдженності значно збільшує об'єм та складність резекції. 
  
  \item \textit{Інвазія пухини в магістральні внутрішньопечінкові судини та жовчні протоки, що потребує їх реконструкції.} Такі втручання можливі в лапароскопічному варіанті, але пов'язані із високою технічною складністю, крововтратою та тривалим часом операції.
  
\end{itemize}

При наявності у хворого відносних протипоказів рішення про виконання операції в відкритому або лапароскопічному варіанті повинно базуватись на оцінці співвідношення потенційних ризиків та користі для пацієнта в кожному конкретному випадку. 

\subsection{Безпосередня підготовка до втручання}

Підготовка до \acrshort{llr} не відрізняється від підготовки до аналогічного відкритого втручання та включає:
\begin{itemize}
    \item отримання згоди на операцію пацієнта або особи, що представляє його інтереси після отримання ними вичерпної інформації про заплановане лікування
    
    \item обмеження прийому їжі з 18 годин попередньої доби
    
    \item обмеження прийому рідини за 6 годин до операції
    
    \item підготовка шкіри шляхом зістригання волосся за допомогою тримера
    
    \item підготовка кішківника шляхом очисних клізм або прийому послаблювальних засобів на основі макрогола
    
    \item зупинки прийому антикоагулянтів та антиагрегантів за 7-10 днів до втручання (після погодження із профільними спеціалістами)
\end{itemize}



\section{Технічне забезпечення лапароскопічних резекцій печінки}

\subsection{Вимоги до лапароскопічної стійки}

\subsection{Інструменти доступу та видалення препарату}

\subsection{Базові інструменти}

\subsection{Зшиваючі аппарати}

\subsection{Хірургічні енергії}

\subsection{Спеціалізоване обладнання}


\section{Етапи лапароскопічної резекції печінки}

\subsection{Укладка пацієнта}

В лапароскопічній хірургії положення пацієнта є ключовим фактором, який впливає на ефективність операції. В умовах, коли тракція органів обмежена можливостями ендоінструментів та об'ємом замкнутої черевної порожнини, позиціонування пацієнта може, за рахунок гравітації, забезпечити необхідну експозицію. Окрім початкової позиції на столі важлива динамічна адаптація нахилу самого хірургічного стола відповідно завданням кожного з етапів втручання. Через значний вплив на технічну складову операції ми вважаємо за доцільне розглядати укладку пацієнта як один з етапів \acrshort{llr}.

\subsubsection{Загальні принципи укладки пацієнта}

Позиція пацієнта повинна відповідати запланованому втручанню та анатомічним особливостям пацієнта. Вибір позиції обумовлює запланована лінія транссекції, яка в оптимальному варіанті повинна знаходитись безпосередньо перед хірургом, в продовженні фізіологічної площини його зору. 

Нахил столу повинен забезпечувати основну гравітаційну тракцію внутрішніх органів, яку хірург лише моделює ендоінтструментами. Такий підхід значно знижує ризик травматичних пошкодженнь. Як правило, після постановки портів стіл переводять в зворотнє положення Транделенбурга, що дозволяє відвести кішківник в нижні відділи черевної порожнини. За потреби ця позиція може бути доповнена ізольованим опущенням головного кінця столу. При цьому необхідно щоб положення попереку пацієнта співпадало по рівню з відповідним шарнірним механізмом столу. Нахил столу вліво чи вправо повинен постійно змінюватись в залежності від потреб поточного етапу операції. 
Анестезіологічна бригада повинна бути попереджена про постіну необхідність таких змін положення операційного столу, так як це може впливати на розташування обладнання. 

Після укладки пацієнт повинен бути прифіксований до столу спеціальними ременями, що йдуть в комплекті. Для профілактики позиційних плекситів необхідно намагатись забезпечити максимально можливе фізіологічне положення кінцівок. Перед накриттям пацієнта стерильною білизною необхідно перевірити можливі місця прилягання до тіла пацієнта твердих деталей столу та захистити ці місця силіконовими подушками з метою попередження утворення пролежнів. Також треба уникати можливого провисання попереку та плеча при латеральній та напівлатеральній позиціях.


\subsubsection{Французька позиція}

При Французькій позиції пацієнт знаходиться в положенні на спині, руки розведені або приведені вздовж тулуба, ноги розведені та фіксовані в фізіологічному напівзігнутому положенні за допомогою спеціальних стремен, які фіксуються до операційного столу. Оператор стоїть між ногами пацієнта, перший асистент та відеоасистент стоять ліворуч від пацієнта. За потреби перший асистент може переходити праворуч від пацієнта. Після постановкі портів стіл переводиться в зворотню положення Тренделенбурга (підйом головного кінця разом із опущенням ніг). Перевагами такої позиції є оптимальне положення оператора відносно лінії транссекції при виконанні резекцій із сагітально розташованою лінією транссекції -  правобічної та лівобічної гемігепатектомій, правобічної задньої секцієектомії та лівобічної латеральної секцієектомії. Також вона зручна для проведення антеролатеральних резекцій. Недоліками такої позиції є близькість розташування оператора та відеоасистента, що може заважати вільним рухам інструментів та відносна незручність положення для виконання конверсії.

\subsubsection{Японська позиція}

Японська позиція передбачає положення пацієнта на спині, руки розведені або приведені вздовж тулуба, ноги приведені. Оператор стоїть з правого боку від пацієнта, перший асистент та відеоасистент стоять ліворуч від пацієнта. Фактично положення в цій позиції аналогічне положенню пацієнта під час відкритої операції. Це дає можливість хірургу, що має досвід відкритої резекційної хірургії швидше адаптуватись до лапароскопічного підходу. Також японська позиція дає дає кращий доступ до новоутворень розташованих в правій долі. При необхідності в конверсії така позиція більш зручна та звична для всього персоналу.

\subsubsection{Інші види позицій}

Альтернативою попередніх двох позицій для доступу до постеролатеральних сегементів були запропоновані ліва латеральна та напівлатеральна позиції. Перевага таких позиції в тому, що за рахунок гравітації вони значно зміщують всю печінку ліворуч, що забезпечує можливість досягнення адекватної експозиції із мінімальною тракцією. Під час укладки пацієнта в ці позиції пацієнт частково чи повністю повертається на лівий бік, а його права рука розміщується на спеціальному фіксаторі над головою. Додаткове опущення головного кінця столу дозволяє збільшити простір для маніпуляцій та розширити міжреберні проміжки. Така позиція дозволяє покращити доступ до виконання Sg 8. Недоліком цих позицій є більший ризик позиційних плекситів, погіршання доступу до лівої долі печінки та незручність під час конверсії.

\subsection{Доступ та розстановка троакарів}

Загальний принцип постановки портів єдиний для всіх лапароскопічних операцій, і полягає в тому, що троакари повинні забезпечувати ефективний до ступ інструментів та лапароскопа для безперешкодних маніпуляцій в місці втручання. Зазвичай розташуванні портів вздовж умовної окружності радіусом в \( \frac{1}{2} \) від робочої довжини інструментів проведеної навколо місця втручання. При цьому кут між робочими троакарами оперуючого хірурга повинен складати 60\degree, а порт для лапароскопа повинен знаходитись посередині між ними. Нажаль досягти такої ідеальної розстановки троакарів краще за все вдається на тренажері. У реальному житті зробити це значно важче через анатомічні особливості кожного конкретного пацієнта, проте основний принцип залишається сталим. В подальших розділах ми наводимо орієнтовні схеми розстановки троакарів, які у більшості пацєнтів показали себе зручними для виконання тих чи інших втручаннь. Варто зауважити, що такі схеми не є абсолютним стандартом, тому нижче ми наводимо загальні рекомендації щодо позиціонування за допомогою яких необхідно адаптувати точки доступу в кожному окремому випадку.

\subsubsection{Методи постановки першого троакару}

Існує декілька методів постановки першого троакару:
\begin{enumerate}
    \item постановка троакара після попереднього створення карбоксиперитонеуму голкою Вереша
    \item постановка троакара без попереднього створення карбоксиперитонеуму захищеним троакаром
    \item постановка оптичного троакару із пошаровою візуалізацією за допомогою лапароскопа
    \item відкрита постановка порта через мінілапаротомію по Хассену
\end{enumerate}
 
За данними літератури немає різниці в рівні інтраопераційних ускладненнь між цими методами при первинному доступі у пацієнта без попереднього анамнеза оперативних втручаннь. У пацієнтів що перенесли раніше відкрите або лапароскопічне втручання найбільш безпечною вважають відкриту мінілапаротомію по Хассену. В таких випадках точку встановлення першого троакару обирають максимально віддалено від попереднього місця втручання та післяопераційного рубця (наприклад в точці Палмера для пацієнтів оперованих на органах малого тазу)

\textit{ \textbf{Техніка виконання мінілапаротомії по Хассену:} В обраному для троакара місці роблять розріз шкіри та підшкірної клітковини довжиною 2-3 см. Далі пошарово розсікають апоневротичні шари, роблячі в них отвори, відповідно до діаметру троакару та розводять м'язи. Після розкриття очеревини черевну порожнину за допомогою пальця перевіряють на предмет наявності спайок поруч з місцем постановки троакара, і, при їх наявності, обережно роз'єднують тупим шляхом на відстані 5-6 см. Далі на края апоневрозу накладають герметизуючий кисетний шов, після чого вводять в черевну порожнину троакар Хассена без стилета. Кисетний шов затягують, канюлю троакара під'єднують до інсуфлятора і створюють рабочий тиск карбоксиперитонеуму. Троакар фіксують окремими швами до шкіри.}


\subsubsection{Позиціонування троакарів}

При виконанні \acrshort{llr} напрямок лінії транссекції паренхіми є ключовим фактором, який визначає місця постановки тороакарів. Для кожного виду резекції існує схема розстановки портів, яка в комбінації із позицією пацієнта дозволяє отримати найкращий доступ. Ця схема повинна задовільняти наступним вимогам, які виконуються вздовж всієї площини резекції:

\begin{itemize}
    \item поле зору лапароскопа може бути направлене вздовж лінії транссекції
    \item один або декілька робочих портів забезпечують положення основного диссекційного інструменту (наприклад ультразвукового диссектора-аспіратора) під кутом 15-20\degree до всіх ділянок площини резекції
    \item з обох сторін від лінії транссекції паренхіми повинні бути передбачені допоміжні порти для тракції. Їх позиція визначається в залежності від величини органу таким чином, щоб забезпечити достатній запас місця для ефективного натягу
    \item інсрумени введені через порти ассистента не пересікаються із основними робочими інструментами та не утруднюють їх рухи
\end{itemize}

\textit{ \textbf{Техніка позиціонування робочих портів:} Щоб задовільнити всі вимоги, позиціонування портів починають із вибору місця для порта лапароскопа ще на передопераційному етапі. За допомогою КТ визначають проекцію площини резекції на передню черевну стінку з урахуванням її розтягнення. При цьому також враховують можливе зміщення печінки після мобілізації та тракції. Після введення першого троакара оглядають черевну порожнину та вводять один із ассистентських троакарів. Далі ультразвуковими ножицями мобілізують круглу та серповидну зв'язки печінки і граспером виконують пробну тракцію печінки. Цей прийом дає можливість оцінити реальне зміщення печінки та обрати оптимальну позицію для порта, через який буде проводитись транссекція. Другий робочий порт встановлюють на відстані не менше 7 см від першого таким чином, щоб забезпечити триангуляцію та оптимальний кут із першим на всій площині транссекції }
 

\subsubsection{Трансплевральний доступ}

При виконанні резекції постеролатеаральних сегментів лінія транссекції розташована у верхніх відділах печінки, тому постановка троакарів через передню черевну стінку може не забезпечувати адекватного доступу. В таких випадках доцільно використовувати трансплевральну постановку робочих портів. 

\textit{ \textbf{Техніка постановки трансплеврального троакару:} В обраній для постановки точці по ІІХ - ІХ  міжребір'ю виконують розріз шкіри відповідно до діаметра троакара, та накладають навколо нього кисетний шов. За допомогою зажима тупим шляхом розводять шари міжреберних м'язів по верхній поверхні нижнього ребра, та роблять отвір в плеврі. Далі троакар на стілеті із захисною надувною манжетою на кінці вводять в плевральну порожнину та проводять крізь діафрагму в черевну порожнину. Манжету роздувають та фіксують троакар до шкіри із натягом, щоб забезпечити герметику черевної порожнини. Після заверження втручання трансплевральний троакар видаляють першим, після чого отвір в діафрагмі ушивають. Залишки газу із плевральної порожинин видаляють за допомогою відсмоктувача, введеного крізь міжреберний троакарний отвір одномоментно із апаратним роздуванням легені. Кисетний шов затягують та видаляють відсмоктувач. }

\subsection{Мобілізація печінки}

Мобілізація печінки є обов'язковим етапом виконання \acrshort{llr}. Для кожного вида резекції потрібен відповідний об'єм мобілізації в залежності від об'єму видаляємої паренхіми. Недостатня мобілізація зв'язкового апарату є грубою помилкою, яка призводить до інтраопераційних ускладненнь. При дотриманні техніки в лапароскопічному варіанті доступна повна мобілізація печінки аналогічна відкритим втручанням. 

\subsubsection{Мобілізація серповидної та круглої зв'язок}

Всі операції на печінці починають з мобілізації серповидної та круглої зв'язок з метою покращення обзору та полегшення маніпуляцій з печінкою 
\textit{ \textbf{Техніка виконання:} Круглу зв'язку висікають максимально близько до пупка та передньої черевної стінки разом із навколишньою жировою тканиною за допомогою ультразвукових ножиць для того щоб її залишки не утруднювали візуалізацію. Більшу частину круглої зв'язки видаляють, залишаючи лише коротку її ділянку 1-2 см біля входу в печінку для тракції.  Серповидну зв'язку пересікають в краніальному напрямку із відступом 1 см від поверхні печінки для збереження можливості для тракції печінки.}

\subsubsection{Мобілізація кавальних воріт}

Мобілізація кавальних воріт печінки необхідна при втручаннях, які включають резекцію постеролатеральних сегментів. Це ліва та права гемігепатектомії, права передня та права задня секцієектомії, мезогепатектомія, резекції 4, 7, 8 сегментів а також каудальна лобектомія.

\textit{ \textbf{Техніка виконання:} Після розсічення серповидної зв'язки широко надсікають очеревину в місці переходу парієтальної очеревини, що покриває печінку в вісцеральну очеревину діафрагми та розділюють коронарну зв'язку. Рихлу сполучну тканину роз'єднують до чіткої візуалізації ділянки передньої стінки \acrshort{ivc} між медіальними краями правої печінкової вени та загального устя лівої та серединної печінкових вен. Далі диссекцію продовжують в залежності від об'єму резекції до візуалізації латерального краю відповідного венозного устя.}

\subsubsection{Мобілізація лівої долі печінки} 

Мобілізація лівої долі печінки є необхідним етапом лівобічної латеральної секцієектомії, лівобічної гемігепатектомії, лівобічної трисекцієектомії, ізольованих резекцій Sg 2, 3, 4  та спігелевої долі.

\textit{ \textbf{Техніка виконання:} Після пересічення круглої, серповидної та коронарної зв'язок ассистент забезпечує тракцію лівої латеральної секції граспером в каудальному напрямку. Оператор розсікає ліву трикутну зв'язку печінки ультразвуковими ножицями, починаючи від лівого краю загального устя середнної та лівої печінкових вен. При необхідності виділяють та лігують ліву діафрагмальну вену. Далі стіл нахиляють праворуч, ассистент виконуує тракцію печінки граспером в напрямку правого піддіафрагмального простору. Після цього оператор виділяє, лігує та пересікає Аранцієву (венозну) протоку котра розташована в складці між спігєлевою долею та лівою латеральною секцією. Цей маневр забезпечує подальший доступ до виділення лівих портальних структур та лівої печінкової вени.
}

\subsubsection{Мобілізація правої долі печінки}

Мобілізація правої долі використовується для виконання правобічної гемігепатектомії, правобічної задньої секцієектомії, правобічної трисекцієектомії, та ізольованих резекції Sg 7 та 8.

\textit{ \textbf{Техніка виконання:} Мобілізацію проводять в три етапи. Під час першого етапу стіл нахиляють ліворуч. Ассистент забезпечує тракцію правої долі ліворуч поступово зміщуючи напрямок тракції краніально. Оператор граспером, який тримає в лівій руці відтягує вниз заочеревинну клітковину, яка покриває нирку, а ультразвуковими ножицями надсікає очеревину та продовжує розділення правої трикутної зв'язки поступово просуваючись вперед на всю довжину інструментів. В ході диссекції правий наднирник обережно відєднують від печінки. Під час другого етапу ассистент виконує каудальну тракцію діафрагмальної поверхні Sg 8 печинки а оператор розсікає верхню частину трикутної зв'язки в дорзальному напрямку на максимально можливу відстань з цієї позиції. Під час третього етапу ассистент виконує латеральну траекцію в ділянці Sg 7 печинки, а оператор поєднує створені на попередніх етапах площини резекції. Маркером завершення диссекції є візуалізація всього правого краю запечінкового сегмента \acrshort{ivc} до рівня правої печінкової вени.  
}

\subsubsection{Мобілізація запечінкового сегменту нижньої порожнистої вени}

Мобілізація запечінкового сегменту \acrshort{ivc} використовується для виконання всіх резекцій до складу яких входить перший сегмент або його паракавальна порція. Це правобічна гемігепатектомія, правобічна задня секцієектомія, резекція Sg 7 печінки.

\textit{ \textbf{Техніка виконання:}  Етап виконують після мобілізації правої долі печінки. Стіл нахиляють наліво, ассистент проводить тракцію правої долі за допомогою граспера та іррігатора-аспіратора. Під час тракції основне навантаження виконують граспером, а відсмоктувачем забезпечують лише додаткову тракцію та санацію операційного поля. Оператор виконує диссекцію за допомогою ультразвукових ножиць та граспера в краніальному напрямку вздовж правого края передньої стінки \acrshort{ivc} поступово виділяючи, лігуючи та пересікаючи короткі печінкові вени. В ході мобілізації виділяють та пересікають між кліпсами праву гепатокавальну зв'язку. Об'єм мобілізації визначають в залежності від запланованого втручання. При виконанні тотальної каудальної лобектомії (ізольовано чи в складі іншої резекції) мобілізують всю передню напівокружність \acrshort{ivc}, у інших випадках достатньо мобілізації правої частини передньої стінки \acrshort{ivc}. Показником завершеної диссекції є чітка візуалізація нижнього краю загального устя лівої та серединної печінкових вен і правої печінкової вени. Останню обходять диссектором та беруть на трималку або заводять турнікет для хенгінг-маневру. 
}

\subsection{Прийом Прінгла}

Прийом Прінгла це спосіб тимчасового перетискання елементів печінково-дванадцятиперсної зв'язки з метою часткової васкулярної ексклюзії печінки. Прийом Прінгла дозволяє знизити об'єм крововтрати за рахунок виключення портального притоку та зниження кровотоку по печінковим венам.

\subsubsection{Покази та загальні правила виконання}

Накладання турнікету на печінково-дванадцятиперсну зв'язку для можливості швидкого перетискання портального притоку є стандартом безпечного виконання резекцій печінки, та повинно бути виконано у всіх випадках. Використання прийому Прінгла не є обов'язковим етапом виконання резекції, проте значно полегшує її виконання та повинно бути швидкодоступним на випадок гострої кровотечі. Перетискання портального притоку використовується інтервалами по 15 хвилин з перервами по 5 хвилин між затисканнями турнікета. Доведений безпечний сумарний час перетискання портальних структур складає 90 хвилин. Для пацієнтів із дифузними змінами паренхіми печінки цей час може зменшуватись. 
Запропоновано інтра- та екстракорпоральний варіанти лапароскопічного виконання прийому. Обидва варіанта передбачають накладання турнікету навколо печінково-дванадцятипалої зв'язки та відрізняються способами його перетискання.

\subsubsection{Інтракорпоральний варіант}

Інтракорпоральний варіант прийому Прінгла може бути виконаний за допомогою ендобульдога або за допомогою турнікета та короткої трубки. 
Переваги: не потребує постановки додаткового порту.
Недоліки: менш ефективний у пацієнтів із надлишковою вагою, процес перетискання займає більше часу, складніше накласти в умовах кровотечі.

\textit{ \textbf{Техніка накладання інтракорпорального прийому Прінгла.} Операційний стіл ротують ліворуч. Ассистент виконує краніальну тракцію печінки ендограспером за круглу зв'язку. Оператор проводить атравматичний граспер з правого  латерального порту та заводить його в отвір Уінслоу. Після цього стіл повертають праворуч та оператор іншим інструментом робить отвір в малому сальнику, візуалізує раніше заведений граспер та проводить ним турнікет довжиною 10-12 см навколо елементів печінково-дванадцятипалої зв'язки. Кінці турнікету фіксують за допомогою пластикової кліпси. Для перетискання портальних структур оператор виконує тракцію елементів печінково-дванадцятиперсної зв'язки вгору граспером за турнікет, а іншою рукою накладає на них за допомогою апплікатора накладає на них затискач типу "ендобульдог".}

\subsubsection{Екстракорпоральний варіант}

Екстракорпоральний варіант прийому Прінгла виконують за до помогою турнікета та трубки виведених через порт.

Переваги: ефективний у всіх пацієнтів, кровоток може бути перетиснутий швидко та в умовах обмеженої видимості.
Недоліки: потребує постановки додаткового порту.

\textit{ \textbf{Техніка накладання екстракорпорального прийому Прінгла.} Турнікет довжиною 60 см проводять навколо елементів печінково-дванадцятиперсної зв'язки аналогічно до інтракорпорального воаріанту. Обидва кінці турнікету виводять назовні через окремий 12 мм порт та проводять крізь поліпропіленовоу трубку для відтискання. Трубку перетискають затискачем для забезпечення герметичності черевної порожнини.}


\subsection{Обробка портальних структур}

Виконання анатомічної \acrshort{llr} передбачає резекцію ділянки паренхіми, яка має ізольоване кровопостачання. Для визначення меж такох ділянки необхідне виділення портальних структур відповідного рівня -- дольових, секційних, сегментарних чи субсегментарних. Існує два основних варіанта обробки портальних стуктур. Це глісоновий підхід та індивідуальне лігування.

\subsubsection{Глісоновий підхід}

Суть глісонового підходу полягає в диссекції портальних структур в складі глісонової ніжки - судинно-секреторного пучка огорнутого сполучною фіброзною тканиною, відповідного порядку. Сучасний підхід передбачає обережну диссекцію в просторі між капсулою Лаенека та глісоновою ніжкою. Це дозволяє виділяти судинно-секреторні пучки включно до рівня субсегментарного порядку. Для обрання точок такої диссекції використовують концепцію "вхідних воріт" - міснь нещільного прилягання капсули Лаенека та глісонової капсули. Мета диссекції - створення та подальше з'єднання простору між двома воротами навколо необхідного глісону з метою накладання турнікету.

Глісоновий підхід може застосовуватись при виконаннії всіх видів анатомічних резекцій. Особливістю глісонового підходу є те, що диссекція окремих структур не проводиться, тому аномальні варіанти анатомії повинні бути виключені в доопераційному періоді.

\textit{ \textbf{Техніка глісонового підходу до диссекції портальних структур} Перед виконанням диссекції виконують холецистектомію та проводять турнікет для виконання прийому Прінгла на випадок кровотечі з печінкової артерії або ворітної вени. Ассистент виконує тракцію печінки в краніальному напрямку граспером за круглу зв'язку. Іншою рукою ассистент забезпечує чисте операційне поле аспіратором\-іррігатором. Оператор виконує контртракцію елементів печінково-\-дванадцятиперсної зв'язки граспером за культю міхурового протока та за допомогою ультразвукових ножиць розсікає вісцеральну очеревину над необхідною парою "воріт". Після потрапляння в диссекційний простір між капсулою Лаенека та глісоном подальша диссекція відбувається граспером, тупим шляхом. Після створення  циркулярного тоннеля, навколо необхідного гліссону проводиться турнікет, та виконується його тимчасове перетискання для визначення демаркаційної лінії. При необхідності виділення глісонів більш високого порядку диссекція продовжується периферійно вздовж виділеного стовбура. По демаркаційній лінії діатермією розмічається лінія резекції. Після проведення транссекеції паренхіми, необхідний гліссон виділяється, лігується та пересікається максимально дистально, для зменшення ризику нецільового ураження структур печінкового залишку.}

При виконанні сегментарних та субсегментарних резекцій може бути застосований транспаренхіматозний варіант гліссонового підходу. Відмінністю такого підходу є те, що його виконання не потребує виділення дольових та секційних глісонових ніжок перед виділенням сегментарних або субсегментарних і, відповідно, не потребує холецистектомії.

\textit{ \textbf{Техніка транспаренхіматозного глісонового підходу} Необхідну глісонову ніжку візуалізують за допомогою інтраопераційного УЗД, при цьому також визначають глибину її залягання і маркують її проекцію на поверхню печінки діатермією. Вздовж наміченої лінії роблять надріз капсули довжиною 1,5-2 см. За допомогою ультразвукового диссектора-аспіратора розділяють паренхіму над глісоновою ніжкою, виділяють її та проводять навколо неї турнікет. Положення турнікета підтвержджують повторним УЗД. Після тимчасового перетискання турнікету намічають лінію резекції вздовж лінії демаркації.}

\subsubsection{Індивідуальне лігування}

Під час індивідуального лігування портальні структури виділяють послідовно, та пересікають окремо. Ця техніка більш складна, ніж глісоноивй підхід, але дозволяє відстежити хід структур та визначити їх можливі аномалії. Індивідуальне лігування може застосовуватись для виділення портальних структур першого та другого порядку (дольові та секційні), тому використовується при виконанні гемігепат-, секцієектомій. Показом для виконання диссекції шляхом індивідуального лігування є відома наявність анамалійної будови портальних структур та виконання донорського забору печінки при родинній трансплантації.

\textit{ \textbf{Техніка індивідуального лігування портальних структур} Перед виконанням диссекції виконують холецистектомію та проводять турнікет для виконання прийому Прінгла на випадок кровотечі з печінкової артерії або ворітної вени. Ассистент виконує тракцію печінки в краніальному напрямку граспером за круглу зв'язку. Іншою рукою ассистент забезпечує чисте операційне поле аспіратором\-іррігатором. Оператор виконує контртракцію елементів печінково-\-дванадцятиперсної зв'язки граспером за культю міхурового протока та за допомогою ультразвукових ножиць розсікає парієтальну очеревину над проекцією необхідних структур. При виділенні правих дольових та секційних структур виконують перипортальну лімфаденектомію. Шляхом послідовної лиссекції ультразвуковими ножицями та діатермокоагуляцією виділяють на протязі, беруть на трималки та тимчасово перетискають необхідні гілки печінкової артерії та ворітної вени. Після розмічення лінії резекції вздовж лінії демаркації виділені портальні структури лігують пластиковими кліпсами або лінійним ендостеплером та пересікають. Жовчні протоки виділяють, лігують та пересікають після проведення транссекції та підтверждження їх анатомічної належності частині печінки, що видаляється.}



\subsection{Транссекція паренхіми}

\subsubsection{Відомі методи транссекції паренхіми}

Транссекція паренхіми є основним етапом \acrshort{llr}. Існує велика кількість технік транссекції з використанням різноманітних пристроїв та енергій. 

Історично, під час становлення \acrshort{llr} застосовувались методи прегемостазу, для яких характерно використання коагулюючих пристроїв (наприклад мікрохвильова абляція, високочастотна коагуляція, біполярна коагуляція та ін.) без виділення окремих стурктур вздовж наміченої лінії резекції перед проведенням розділення паренхіми. Транссекція при цьому відбувалась в попередньо прокоагульованій площині з метою зменшення крововтрати. Порівняно із сучасним підходом такі методики мають більший ризик післяопераційнихи ускладнень пов'язаних із краєвим некрозом резекційної поверхні і тому зараз практично не використовуються.

Сучасні методи транссекції паренхіми передбачають два основні варіанти. Це краш-клемпінг та аппаратна диссекція з використанням ультразвукового диссектора-аспіратора чи водяного скальпеля.

\subsubsection{Краш-клампінг}

Розсічення паренхіми печінки за допомогою затискача має наву краш-клампінг або келіклазія (назва походить від назви затискача Келлі). Такій метод є золотим стандартом транссекції паренхіми. В лапароскопічному варіанті затискач Келлі замінюють диссектором Меріленд. Ефективність такої транссекції не поступається іншим інструментальним методам, проте більш складна та тривала. Некорректне використання краш-клампінгу може призводити до травмування внутрішньопечінкових судинних структур великого розміру та кровотечі з них. Наявність диффузних змін паренхіми печінки також може значно затруднювати застосування методики.

\textit{ \textbf{Техніка транссекції паренхіми методом краш-клампінгу.} Під час транссекції невеликі порції паренхіми розміром 1-2 мм оператор розчавлює браншами диссектора. Ассистент за допомогою від\-смок\-ту\-ва\-ча-\-іррігатора забезпечує чистоту поля та візуалізацію змиваючи та евакуюючи залишки паренхіми. Дрібні структури пересікають за допомогою ультразвукових ножиць. Більші структури обходять диссектором, кліпують та лігують. Пройдену поверхню коагулюють біполяром}

\subsubsection{Транссекція за допомогою ультразвукового диссектора-аспіратора}

Ультразвуковий диссектор-аспіратор є ефективним інструментом, який дозволяє безпечно виділяти елементи строми під час транссекції паренхіми печінки. Оперуючий хірург повинен розуміти принцип роботи та налаштування пристрою, мати досвід застосування його у відкритій резекційній хірургії печінки. 

Основним діючим елементом робочої рукоядки аспіратора є сонотрод, який шляхом поступово-зворотніх рухів на великій частоті передає енергію з п'єзоелемента на тканину. Таким чином під час контакту паренхіми печінки із торцем інструменту відбувається її руйнування та аспірація через просвіт сонотроду. При контакті із боковою поверхнею сонотрода відбувається нагрівання тканини за рахунок тертя та ефект коагуляції подібний до дії ультразвукових ножиць. 

Таким чином дія ультразвукового дисектора-аспіратора залежить від місця контакту із тканиною. При торцевому контакті тканина руйнується, при боковому -- коагулюється. Розуміння цього важливе для корректного використання приладу. Під час обробки тонкостінних стурктур, таких як печінкові вени та жовчні протоки необхідно запобігати торцевому контакту їх стінок до інструмента. Забезпечити це можливо шляхом зміни напрямку прикладання зусилля до інструмента. Також ефективним проведення транссекції паренхіми в напрямку від устя печінкових вен до периферії.

\textit{ \textbf{Техніка транссекції ультразвуковим диссектором-\\аспіратором.} Під час транссекції паренхіми хірург рухає кінчик інструмента по веретеноподібній траекторії невеликої амплітуди. Контакт із паренхімою відбувається на зворотньому шляху інструменту, із поступовим нарощенням сили контакту. Прямий торцевий контакт є допустимим лише в безсудинних ділянках паренхіми. Рухаючи інструмент таким чином хірург виділяє елементи строми, кліпує їх та лігує за допомогою ультразвукових ножиць. Застосування такої техніки запобігає травмуванню печінкових вен та зменшує кровотечу. Під час роботи, між етапами активного використання необхідно постійно промивати робочий канал аспіратора фізіологічним розчином для запобігання його блокування залишками такнин}

\subsubsection{Обробка магістральних внутрішньопаренхіматозних структур}

В процесі трассекції всі крупні внутрішньопаренхіматозні структури сегментарного та субсегментарного рівня повинні бути повністю виділені напротязі, обійдені диссектором та ідентифіковані перед лігуванням та пересіченням. Спосіб лігування обирають за розміром структури. Як правило для лігування притоків печінкових вен сегментарного рівня використовують великі металеві кліпси, а для лігування глісонових гілок та магістральних печінкових вен - пластикові кліпси із замком або ендостеплери із судинною касетою. Пересічення структур допускається проводити ультразвуковими ножицями. 

Важливо не починати виділення магістральних судин на прикінці періоду перетискання портального притоку, так як така маніпуляція пов'язана із ризиком кровотечі яка може потребувати подовження часу прийому Прінгла. 

\subsubsection{Тракція печінки під час транссекції}

Ефективна та безпечна транссекція паренхіми можлива лише за умови ефективної тракції та контртракції печінки. В умовах недостатньо натягнутої паренхіми хірург постійно стикається поганою візуалізацією, підвищеним ризиком травмування магістральних внутрішньопечінкових структур та посиленям кровотечі. Адекватна тракція під час виконання \acrshort{llr} є ключовим моментом, що впливає на безпеку та швидкість втручання. При обширних резекціях забезпечення необхідної тракції є складним процесом, що потребує високого рівня хірургічних навичок як хірурга так і асистента. Основні принципи забезпечення ефективної тракції та контртракції наступні:
\begin{enumerate}
    \item положення пацієнта та нахил операційного столу націлені на створення максимальної гравітаційної контртракції печінкового залишку. Це забезпечується повортотом столу в протилежний резекції бік: при лівобічних резекціях праворуч, а при правобічних - ліворуч.
    \item оператор та ассистент виконують додаткову інструментальну тракцію, одномоментно кожен в свій бік, щоб забезпечити максимальне натягнення паренхіми саме в місці транссекції
    \item для зменшення травматизації паренхіми для тракції використовують культю круглої зв'язки або жовчний міхур, який залишають частково фіксованим до печінки після холецистектомії. При відсутності такої можливості у якості трималок використовують окремі шви поліфіламентним матеріалом по краю паренхіми
    \item в заключній третині транссекції можливе використання хенгінг-маневру, суть якого полягає в проведенні турнікету під печінкою вздовж площини транссекці. Виконуючи цей маневр необхідно пам'ятати про те, що він деформує паренхіму та може призводити до зміщення площини резекції
\end{enumerate}

\textit{ \textbf{Техніка виконання хенгінг-маневру за допомогою катетера Фолея.} Найбільш зручний на думку авторів варіант маневру передбачачає використання в якості турнікету сечового катетера Фолея. За рахунок еластичності застосування такого катетера безпечно з точки зору пошкодженнь паренхіми. Для виконання маневру катетер вкорочують до 12-15 см, залишаючи кінець з манжетою. Після проведення під печінкою обрізаний кінець катетера проводять в отвір на іншому його кінці. Отриману кільцеву петлю затягують навколо паренхіми та фіксують пластиковою кліпсою, за яку виконують тракцію. Перевагою такого способу є те, що він надає можливість по ходу транссекції паренхіми звужувати довжину петлі навколо неї та тримати турнікет в постійному натязі.}

Окрім наведеної вище техніки існує варіант так званої маріонеточної техніки, коли трималки, підшиті до паренхіми, виводять кожну окремо назовні через передню черевну стінку. На думку авторів такий варіант пов'язаний з утрудненим доступом до резекційної поверхні та погіршанням тракції в заключних етапах транссекції \cite{Hsu2012}.

\subsection{Евакуація препарата}
Для евакуації видаленого препарата з черевної стінки в обов'язковому порядку повинен використовуватись ендоконтейнер. В наявності повинні бути контейнери всіх розмірів від 300 до 2000 мл. Типово препарат видаляють через невеликий (розміром відповідно до діаметра пухлини) доступ по Пфаненштілю. При наявності у пацієнта рубців від попередніх втручаннь можливо виконання доступу по попередньому рубцю. При необхідності евакуації дрібних макропрепаратів, наприклад після енуклеорезекції, використовують розширення троакарного доступу.

\textit{ \textbf{Техніка евакуації препарата через доступ по Пфаненштілю.} Стіл переводять в положення Тренделенбурга. Шкіру розсікають горизонтально по лінії що поєднує верхні передні ості клубової кістки на мінімально необхідну для видалення препарата довжину (діаметр новоутворення збільшений на 1 см). Шкіру разом із підшкірною клітковиною сепарують від апоневрозу в краніальному напрямку. Розріз апоневрозу проводять вертикально вздовж серединної лінії розміром на 1-2 см більшим за розріз шкіри. М'язи відводять ретракторами тупим шляхом. Після цього в черевну порожнину під контролем відеокамери вводять 12 мм порт, через який заводять ендоконтейнер. Стіл повертають в зворотнє положення Транделенбурга. Макропрепарат занурюють в ендоконтейнер за допомогою ендограсперів та підтягують до розрізу. Карбоксиперетонеум випускають. Задній листок апоневрозу та очеревина досікають на необхідну довжину, ендоконтейнер вилучають з черевної порожнини. Розріз ушивають пошарово, поновлюючи герметику черевної порожнини. Повторно встановлюють карбоксиперитонеум та проводять ревізію черевної порожнини та місця резекції.}


\section{Відхилення від нормального ходу операції та способи їх коррекції}

\subsection{Інтраопераційні ускладнення та технічні сладнощі}

Загальний рівень морбідності для \acrshort{llr} складає 18,3\% за данними метааналізу \cite{Ciria2016a}. Цей показник нижчий за аналогічний (29,7\%) для відкритих операцій через зменшену операційну травму, крововтрату ранню активацію пацієнтів. Не дивлячись на це під час \acrshort{llr} в невеликому відсотку випадків можуть виникати важкі ускладнення та відхилення, які пов'язані із погіршенням ранніх та віддалених результатів операції. Далі ми приведемо механізм розвитку, методи профілактики та коррекції найчастіших з цих ускладненнь.  

\subsubsection{Кровотеча}

Гостра кровотеча під час виконання \acrshort{llr} є грізним ускладненням, яке є причиною до 69\% конверсій \cite{Michele2021}.  Особливостями такої кровотечі є те, що вона відбувається в обмеженому просторі черевної поржнини. При цьому джерело кровотечі часто знаходиться у вузькій шілині між резекційними поверхнями або між печінкою та діафрагмою. Це призводить до швидкого заповнення операційного поля кров'ю, що суттєво утруднює візуалізацію та ідентифікацію джерела кровотечі. При активній евакуації крові за допомогою аспірації разом із кров'ю евакуюється карбоксиперитонеум, знижується внутрішньочеревний тиск та зникає робочий простір, що ще більше посилює крововтрату. 

Зазвичай гостра кровотеча виникає внаслідок основних трьох причин описаних нижче, або їх комбінації:

\begin{itemize}
    \item \textbf{Розрив пухлини} Деякі з пухлин та доброякісних новоутвореннь печінки, наприклад такі, як \acrshort{hcc}, \acrshort{ihcc} та \acrshort{hmg} є гіперваскулярними. При досягненнями ними великих розмірів та ураженні поверхні печінки є ризик їх розриву під час мобілізації печінки внаслідок інструментальної тракції. При появі такого розриву виникає інтенсивна кровотеча з пухлини, що важко піддається зупинці. Як правило такий розрив є показом до конверсії не тільки через кровотечу, але і через підвищений ризик диссемінації пухлини та погіршання онкологічного прогнозу. 
    
    \textit{Для попередження} цього ускладнення необхідно зважено підходити до виставлення показів к \acrshort{llr} у пацієнтів з великими гіперваскулярними утвореннями та обережно і під візуальним контролем виконувати інструментальну тракцію печінки.
    
    \item \textbf{Пошкодження магістральних портальних структур} може виникати як під час виконання глісонового підходу так і під час індивідуального лігування чи в процесі транссекції паренхіми. Грубі спроби диссекції в невірній площині призводять до пошкодження структур, які важко ідентифікувати та відновити. Окрім кровотечі такі дії можуть призводити до пошкодження судин та жовчних шляхів печінкового залишку і виникнення важких післяопераційних ятрогенних ускладненнь у хворого. 
    
    \textit{З метою попередженння} пошкодження портальних структур необхідно постійно дотримуватись цих правил:
    \begin{enumerate}
        \item Завжди проводити турнікет навколо \acrshort{pdl} та забезпечувати можливість швидкого виконання прийому Прінгла перед диссекцією портальних структур 
        \item Ретельно вивчати в передопераційному періоді анатомічне взаєморозташування портальних структур за данними КТ та МРТ у кожного пацієнта, та забезпечити можливість демонстрації цих дослідженнь в операційній
        \item Верифікувати та дотримуватись в кожному випадку вірної площини диссекції - капсули Лаенека при глісоновому підході та стінки ворітної вени при ізольованому лігуванні
        \item Не проводити агресивних проникаючих рухів диссектором "в сліпу", без адекватної попередньої диссекції та достатньої контртракції навколишніх структур
        \item Під час проведення транссекції паренхіми не потрібно виконувати диссекцію крупних структур в кінці інтервалу перетискання портального притоку 
    \end{enumerate}
    
    \item \textbf{Пошкодження магістральних печінкових вен} є джерелом найбільш небезпечних та інтенсивних кровотеч, які важко піддаються зупинці \cite{Li2016}. Це пов'язано із анатомічно важкодосяжним інтрапаренхіматозним положенням печінкових вен (\acrshort{hv}) та малою товщиною їх стінок. Інтенсивність кровотечі з \acrshort{hv} залежить від розміру дефекту та різниці тисків карбоксиперитонеуму в черевній порожнині та крові в просвіті вени. 
    
    \textit{Для попередження} пошкодження стінок магістральних \acrshort{hv} та зменшення кровотечі з них можливо впливати на кожен з цих факторів:
    
    \begin{enumerate}
        \item Зменшення розміру потенціального дефекту \acrshort{hv} можна досягти шляхом проведення диссекції ультразвуковим диссектором-аспіратором в напрямку від устя до периферичних відділів \acrshort{hv} . Така диссекція дозволяє йти анатомічно вздовж стінки судини, та у випадку пошкодження судинної стінки додатково не збільшувати дефект \cite{Honda2020}.  
        
        \item Короткочасне підвищення тиску карбоксиперитонеуму до 15-20 мм.рт.ст. може бути ефективним в в окремих випадках для зупинки кровотечі з \acrshort{hv}, проте цей прийом пов'язаний із підвищеним ризиком газової емболії \cite{Otsuka2013, Cassai2019} та повинен бути застосований з великою обережністю.
        
        \item Зниження тиску в печінкових венах є найбільш ефективним шляхом зниження загальної крововтрати під час \acrshort{llr}. Його можливо досягнути проведенням так званої low-CVP анестезії та зниженням тиску в дихальних шляхах. Зниження центрального венозного тиску досягається за рахунок використання газового наркозу та обмеження перед- та інтраопераційного рідинного навантаження пацієнта. Зниження пікового тиску в дихальних шляхах досягається переведенням пацієнта в режим вентиляції з керуванням по інспіраторному тиску. В окремих випадках у толерантних пацієнтів можлива зупинка вентиляції на видиху на термін 60 - 90 секунд для візуалізації хірургом джерела кровотечі \cite{Kobayashi2016}.
    \end{enumerate}
    
\end{itemize}

Також до масивної крововтрати може призводити незначна за темпом, але тривала кровотеча з декількох джерел. Така ситуація може виникати при швидкому просуванні транссекції паренхіми без приділення необхідної уваги гемостазу. При цьому темп кровотечі додатково посилюється між періодами виконання прийому Прінгла. Така тривала кровотеча призводить до погіршання візуалізації через постійне забруднення операціного поля згустками крові та до втрати часу на його регулярну санацію. 


\subsubsection{Перфорація порожнистого органа}

Виражений злуковий процес, необережне застосування хірургічних енергій або використання інструментів з дефектами ізоляції можуть бути причиною пошкодження навколишніх порожнистих органів під час проведення \acrshort{llr}. Найбільша небезпека таких пошкодженнь в тому, що вони можуть лишатись непоміченими під час операції та призводити до важких ускладненнь в післяопераційному періоді. 

\textit{З метою попередження} перфорації порожнинного органа відеоасистент повинен супроводжувати всі активні рухи інструментів як оператора так і першого асистента. Після постановки першого троакара обов'язково необхідно оглянути простір під ним на предмет ураженнь, а також оглянути сам троакар завівши лапароскоп через інший порт. Активація хірургічної енергії повинна виконуватись тільки після досягнення достатнього контакту із тканиною, а тракцію під час активації необхідно спрямовувати в безпечному напрямку. Всі інструменти повинні бути перед кожною операцією оглянуті на предмет явних пошкодженнь ізоліяції та перевірені на спеціальному обладнанні. 


\subsubsection{Перфорація діафрагми}

Порушення цілосності правого куполу діафрагми можливе при мобілізації правої долі печінки при пухлинній інвазії діафрагми. При цьому вуглекислий газ з черевної порожнини потрапляє в праву плевральну порожнину і тиск між ними вирівнюється. Через деякий час це призводить до погіршання дихальної функції легені, збільшення карбоксемії та гемодинамічної нестабільності через змішення середостіння. Окрім того за відсутності негативного тиску в плевральній порожнині правий купол діафрагми починає "парусити", що утруднює доступ до верхніх відділів правої долі печінки.

\textit{Для відновлення} цілісності діафрагми та негативного тиску в плевральній порожнині дефект ушивають шовним матеріалом з насічками (V-loc або STRATAFIX). Перед затягуванням останнього шва з плевральної порожнини евакуюють весь газ за допомогою аспіратора. Якщо цей маневр не вдається ефективно виконати праву плевральну порожнину дренують по Бюлау.

\subsubsection{Помилкова ідентифікація структур}

Особливістю лапароскопічних операцій є те, що кут огляду значно вужчий, а поле зору відповідно значно менше, ніж при відкритих операціях. Окрім того, як і всі оптичні прилади, лапароскоп дає часткове спотворення перспективи. Ці особливості можуть призводити до втрати анатомічної орієнтації і помилкової ідентифікації структур під час \acrshort{llr}. Найбільш частою проблемою є невірна оцінка розміру структури. Збільшення, яке надає лапароскоп призводить до того, що відносно малі структури можуть ідентифікуватись хірургом, як магістральні. Також до невірної ідентифікації може призводити наявність у пацієнта варіантної анатомії, злуковий процес в ділянці воріт печінки, зміни після раніше проведених черезшкірних абляцій. Помилкова ідентифікація та лігування структур небезпечна порушенням кровообігу та жовчевідтоку в потенційному печінковому залишку та асоційована із розвитком важких ускладненнь - печінкової недостатності, некрозу та абсцедування частини печінки, жовчних нориць та стриктур.

\textit{Для попередження} помилкової ідентифікації структур необхідно дотримуватись наступних правил:
\begin{enumerate}
    \item Ретельно вивчати передопераційну анатомію на предмет можливих варіантів будови
    \item При зміні кута огляду завжди виконувати повторний панорамний огляд операційного поля з ідентифікацією основних відомих орієнтирів (елементів воріт печінки, жовчного міхура, круглої зв'язки печінки тощо) 
    \item Для оцінки розміру структур порівнювати їх із елементами відомого розміру (інструментами, кліпсами та ін.)
    \item Перед пересіченням структури виділяти її на достатній для прослідковування її ходу відстані
    \item Не лігувати магістральні структури в умовах поганої візуалізації або кровотечі
\end{enumerate}


\subsubsection{Зміщення із запланованої резекційної площини}

Через те, що одномоментний огляд всієї печінки з одного положення неможливий або затруднений під час проведення \acrshort{llr} хірург як правило бачить лише частину органа. Окрім того, при проведенні великих анатомічних резекцій, таких як гемігепатектомія, секцієектомія чи мезогепатектомія, на заключних етапах транссекції через конституцію черевної порожнини асистенту стає значно важче забезпечити адекватну тракцію в необхідному напрямку. Ці фактори можуть призводити до неконтрольованого зміщення площини транссекції, що призводить до неанатомічних або R1 резекцій та погіршує онкологічний результат. 

\textit{Для ефективного контролю} за дотриманням вірної площини транссекції необхідно:
\begin{enumerate}
    \item забезпечити повну та адекватну обраному об'єму резекції мобілізацію печінки з досягненням чіткої візуалізації та ідентифікації усіх ключових структур
    \item повністю намітити лінію резекції як на вісцеральній так і на діафрагмальній поверхні печінки
    \item під час розмічення лінії резекції і під час подальшої транссекції паренхіми рутинно використовувати інтраопераційну УЗ-діагностику з визначенням місця знаходження внутрішньопечінкових орієнтирів
    \item впродовж проведення трансекції паренхіми регулярно проводити панорамний огляд печінки з орієнтуванням по положенню магістральних структур
    \item при виконанні хенгінг-маневру враховувати деформацію паренхіми та латеральне змішення ліні транссекції які при цьому відбуваються
\end{enumerate}

\subsection{Конверсія на відкрите втручання}

\subsubsection{Процес прийняття рішення про конверсію}

За сучасними уявленнями конверсія на відкрите втручання сама по собі є не ускладненням, а безпечним способом завершити втручання \cite{AbuHilal2017a}. Загальний ризик конвесії варіює в діапазоні  2 - 21,7\% та залежить від досвіду центру та показів до \acrshort{llr} \cite{Ciria2016b}. 

Розрізняють два види причин переходу з лапароскопічного на відкриті втручання: причини пов'язані із несприятливими знахідками та пов'язані із несприятливими подіями під час операціями. До \textit{несприятливих знахідок} відносять конверсії внаслідок вираженого злукового процесу, неможливості забезпечення онкологічної адекватності (перед початком резекції), поганого доступу, поганої якості паренхіми печінки, анатомічних аномалій. До \textit{несприятливих подій} відносять конверсії пов'язані із гострою кровотечею, неможливістю забезпечення онкологічної адекватності (під час резекції), ураженням оточуючих структур, гемодинамічною нестабільністю. Такий поділ пов'язаний із тим, що результати конверсій внаслідок несприятливих подій достовірно погіршують результати лікування в порівнянні з конверсіями внаслідок несприятливих знахідок \cite{Halls2018}. До чинників, асоційованих із конверсіями автори дослідження відносять неоадьювантну хіміотерапію, попередньо перенесену резекцію печінки, обсях резекції та локалізацію ураження в погано доступних сегментах.

Вчасне прийняття рішення про конверсію - важливий елемент безпеки пацієнта та завжди є стресовим моментом для хірургічної бригади. Враховуючи це, для зменшення впливу суб'єктивних факторів, критерії конверсії повинні бути визначені в передопераційному періоді індивідуально в кожному випадку та обговорені із хворим. Так, наприклад, у пацієнтів, що мають зазначені вище фактори ризику, необхідно визначати гранично припустимий об'єм крововтрати при досягненні якого виконувати конверсію. Також кожен хірург виходячі із оцінки власного досвіду та рівня технічних навичок повинен визначити власні критичні моменти оперативного втручання, які для нього є абсолютними показами для конверсії. Цей перелік може включати кровотечу з магістральної печінкової вени, інвазію пухлини в портальні стуктури першого порядку, інвазію пухлини в перший сегмент та інше. Такий підхід дозволяє зробити процес прийняття рішення більш об'єктивним, безпечним і дати чіткий алгоритм дій в екстренній ситуації.

\subsubsection{Технічні аспекти виконання конверсії}

Можливість екстренного переходу з лапароскопічного на відкрите втручання повинна бути передбачена для кожного втручання. Хірургічна бригада, включаючи медичних сестер та анестезіологів повинна бути інформована про порядок дій під час конверсії. Конверсія включає наступні етапи:
\begin{enumerate}

    \item \textbf{Досягнення тимчасового гемостазу} При конверсії з приводу гострої кровотечі обов'язково необхідно досягнути тимчасового гемостазу шляхом накладання прийому Прінгла та притискання місця кровотечі граспером або серветкою. Ігнорування цього правила може призводити до масивної крововтрати та гемодинамічної декомпенсації пацієнта під час проведення лапаротомії. Важливо розуміти, що зняття карбоксиперитонеуму посилює кровотечу, а наявність згустків крові в операційному полі значно утруднює пошук її джерела.
    
    \item \textbf{Переведення пацієнта в положення на спині.} В разі використання при проведенні \acrshort{llr} позицій відмінних від японської при проведенні планової конверсії пацієнта необхідно перемістити в звичайне положення на спині. Це забезпечує зручність та безпеку подальшого втручання. Напівлатеральна та латеральна позиції утруднюють лапаротомний доступ до печінки, але разом з тим на зміну позиції необхіден певний час. Враховуючи це, необхідність переукладання пацієнта під час ургентної конверсії визначається в залежності від темпу кровотечі та його гемодинамічного стану. 
    
    \item \textbf{Вибір місця та об'єму лапаротомного доступу.} Лапаротомія повинна забезпечувати адекватну експозицію операційного поля для проведення всіх подальших етапів втручання. При виконанні ургентної конверсії використовують максимально широкі доступи типу "мерседес" або J-доступ, аналогічно до \acrshort{olr}. При проведенні планової конверсії з метою транссекції паренхіми після виконання попередніх етапів операції в лапароскопічному варіанті можливо обмежитись невеликим підреберним розрізом.
    
    \item \textbf{Заміна набору інструментів} Базовий набір відкритих хірургічних інструментів повинен постійно знаходитись в доступі. Це дозволяє хірургу швидко почати виконувати лапаротомію, поки операційна сестра проводить заміну основного набору. В оптимальному варіанті повністю укомплектований стерильний набір інструментів для відкритої операції викладається під час кожної \acrshort{llr} на окремому столику Мейо. Під час проведення конверсії цей столик замінює основний робочий столик операційної сестри, що значно скорочує час виконання конверсії.
    
\end{enumerate}


\printbibliography [heading=subbibliography]
\end{refsection}
\chapter{Етапи лапароскопічних резекцій печінки}
\begin{refsection}
\section{Позиція пацієнта}

\subsection{Французька позиція}

\subsection{Японська позиція}

\subsection{Інші види позицій}

\section{Доступ та розстановка троакарів}

\subsection{Методики постановки першого троакару}

\subsection{Позиціонування троакарів}

\subsection{Трансплевральний доступ}

\subsection{Торакальний трансдіафрагмальний доступ}

\section{Мобілізація печінки}

\subsection{Мобілізація кавальних воріт}

\subsection{Мобілізація лівої трикутної зв'язки}

\subsection{Мобілізація правої трикутної зв'язки}

\subsection{Мобілізація запечінкового сегменту нижньої порожнистої вени}

\section{Прийом Прінгла}

\subsection{Загальні правила виконання прийому}

\subsection{Екстракорпоральний варіант}

\subsection{Інтракорпоральний варіант}

\section{Обробка портальних структур}

\subsection{Глісоновий підхід}

\subsection{Індивідуальне лігування}

\section{Транссекція паренхіми та методи гемостазу}

\subsection{Методи транссекції паренхіми та їх порівняння}

\subsection{Краш-клампінг}

\subsection{Транссекція за допомогою кавітаційного ультразвукового диссектора-аспіратора}

\subsection{Лігування крупних структур}

\section{Евакуація препарата}

\section{Типові помилки та їх вирішення}

\printbibliography [heading=subbibliography]
\end{refsection}